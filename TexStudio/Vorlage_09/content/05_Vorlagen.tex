% Vorlagen
%
Dieses Kapitel enthält zahlreiche Vorlagen, die beim Erstellen eines \LaTeX "= Dokuments von grossem Nutzen sein können. Grundsätzlich können die hier enthaltenen Codestellen stets kopiert und entsprechend angepasst werden. Bei Fragestellungen, welche über die hier aufgeführten Punkte hinausgehen, ist das Internet zu konsultieren.%
%
\section{Leerzeilen und Absätze}%
%
Eine neue Zeile beginnt man mit dem nachfolgenden Befehl (vgl. Code) \\
Eine Leerzeile fügt man so ein: (vgl. Code) \\ \\
Wenn man nach einem Lauftext im Code eine Leerzeile einfügt, erstellt dies nur eine neue Zeile. (vgl. Code)

Wenn danach ein section, subsection oder subsubsection-Befehl kommt (wie nach diesem Abschnitt) hat die Code-Leerzeile keinen negativen Einfluss auf das Layout. (vgl. Code) 

\subsubsection{Subsubsection}

Des Weiteren haben auch mehrere aufeinanderfolgende Leerzeilen im Code keinen (zusätzlichen) Einfluss auf den Output.






Wie hier zu sehen ist.
%
\subsubsection{Wichtig!}
%
Am besten wird aber wie in diesem Codefile vorgezeigt auf das Einfügen von Zeilenumbrüchen durch Leerzeilen im Code verzichtet. Leerzeilen, welche den Code übersichtlicher gestalten sollen, werden vorteilhaft mit einem \% "=Zeichen (auskommentieren) ausgefüllt, damit sie der Compiler ignoriert und somit keine unerwünschten Nebeneffekte auftreten.
%
\subsubsection{Achtung}%
%
Es wird empfohlen, am Ende jedes Abschnittes und jedes Befehl"=Aufrufs im Code ein Auskommentier"=Zeichen (\%) zu setzen. (vgl. Code)%
%
Mit dieser Massnahme wird erreicht, dass der Compiler den \gf{Enterschlag} am Ende der Codezeile ignoriert. Dieser Enterschlag wird vom Compiler als \gf{Leerschlag} interpretiert. In 99.9\% hat dieser Leerschlag keinen Layout"=bestimmenden Einfluss. In sehr seltenen Fällen (wenn bei einem Abschnitt im PDF die letzte Linie vor dem Umbruch gerade voll wird) kann eine komplett leere Linie entstehen. Beim Aufruf von Befehlen (z.B. Einfügen eines Bildes mit \gf{\textbackslash spic}) kann u.U. am Anfang eines Absatzes ein Leerschlag resultieren, was wie ein (ungewünschtes) \gf{Einrücken} des Abschnitts aussieht. 
%
%
%
%
%
\section{Formeln}%
%
Nachfolgend sind verschiedene Arten beschrieben, wie eine Formel integriert werden kann.\\
Es ist sehr wichtig, dass man bei den Enter"=Schlägen im Code das untenstehend angewendete Vorgehen genau nachahmt, so dass die Abstände zwischen Text und Formeln stimmen. (allenfalls Leerbereiche mit \%-Zeichen einfügen, so dass die Zeile komplett auskommentiert wird).%
%
Eine Formel%
%
\begin{equation}
     \epsilon_{c\sigma}(t)=\int_0^t J(t,\tau) d\sigma_c(\tau) \label{Formellabel1}
\end{equation}
%
% Wenn man Leerzeilen einfügt und diese mit Kommentaren füllt, hat dies keinen Einfluss auf das Layout.
%
Zwei Formeln mit separater Nummerierung%
%
\begin{align}
 &f_1(x) = (x+a)(x+b) \label{Formellabel2} \\
 &f_{221}(x)= x^2 + (a+b)x + ab \label{Formellabel3}
\end{align}
%
Zwei Formeln mit separater Nummerierung und Ausrichtung beim Gleichheitszeichen%
%
\begin{align}
 f_1(x) &= (x+a)(x+b) \label{Formellabel4} \\
 f_{221}(x)&= x^2 + (a+b)x + ab \label{Formellabel5}
\end{align}
%
Zwei Formeln mit einer Nummerierung%
%
\begin{equation}
\begin{gathered}
\begin{split}
 &t_T = \sum\limits^{n}_{i=1} \Delta t_i \exp{\left(13.65-\frac{4000}{273+ T(\Delta t_i )}\right)} \\
 &f_{221}(x)= x^2 + (a+b)x + ab 
\end{split}
\end{gathered}
\label{Formellabel6}
\end{equation}
%
Zwei Formeln mit Subequation, a und b%
%
\begin{subequations} 
\begin{align} 
\renewcommand\theequation{\theparentequation.\alph{equation}} 
&f_1(x) = (x+a)(x+b)  \label{Formellabel8}\\
&f_{221}(x)= x^2 + (a+b)x + ab \label{Formellabel9}
\end{align}
\end{subequations} 
%
Mehrere Gleichungen pro Zeile mit wählbarer Position der hinteren Gleichungen und einer Nummer pro Zeile%
%
\begin{alignat}{2} % Zahl entspricht Anzahl der Spalten
&y = x^2 + bx + c     \hspace{29mm} && f(x) = x^2 + 2xy + y^2+2xy   \label{Formellabel10}\\
&y = ax^2 + bx + c + d              &&  \left(\frac{35}{f_{cc}}\right)^{0.2}     \label{Formellabel11}\\
&y = ax^2 + c                       && \theta = \frac{1}{2} \, \arccot \left( \frac{\epsilon_y - \epsilon_x}{\gamma_{xy}} \right)   \label{Formellabel11b}
\end{alignat}
%
Mehrere Gleichungen pro Zeile mit nicht wählbarer Position der hinteren Gleichungen und einer Nummer pro Zeile%
%
\begin{align}
&f_1(x) = (x+a)(x+b)                   &  &w =z  \label{Formellabel12} \\            
&f_{221}(x)= x^2 + (a+b)x + ab         &  &3w=\frac{1}{2}z \label{Formellabel13}      
\end{align}
%
Mehrere Gleichungen pro Zeile mit nicht wählbarer Position der hinteren Gleichungen und einer Nummer pro Zeile, inklusive einer Zeile ohne Nummer%
%
\begin{align}
&f_1(x) = (x+a)(x+b)           &  &w =z2x=-y  \label{Formellabel15} \\            
&f_{221}(x)= x^2 + (a+b)x + ab         &  &3w=\frac{1}{2}z2x=-y  \notag\\
&-4 + 5x=2+y   &  &w+2=-1+w2x=-y   \label{Formellabel16}         
\end{align}
%
Mehrere Zeilen mit mehreren Gleichungen mit nur einer Formelnummer%
%
\begin{equation}
\begin{aligned}
&f_1(x) = (x+a)(x+b)    \hspace{45mm}       &  &w =z  \\  
&2x=-y+\alpha_H                         &  &3w=\frac{1}{2}z 
\end{aligned}
\label{Formellabel17}
\end{equation}
%
Formeln im Lauftext werden folgendermassen geschrieben: (vgl. Code) $f_1(x) = (x+a)(x+b)$. Es empfiehlt sich, auch alle Variablen, die man erwähnt, auf diese Weise im Text einzufügen, zum Beispiel $A_c$ oder $E_{c,28}$. Auf diese Weise wird gewährleistet, dass die Formatierung in jedem Fall korrekt ist. Brüche im Text stellt man entweder so: $\sfrac{b+c}{\alpha}$, oder so $\frac{b+c}{\alpha}$ dar.%
%
\section{Bild einfügen (und referenzieren)}%
%
Ein Bild fügt man mit den dafür definierten Befehlen aus dem Style-File ein. Darauf verweisen kann man mit dem folgenden Befehl, der hier auf das \autoref{testBild} weiterleitet. Bei Gleichungen wird folgender Befehl angewendet: \eqref{Formellabel16}. Die mit diesem Befehl eingefügten Bilder haben die Breite 150mm. (für den Bild-Einfüge-Code bitte Code konsultieren)\\%
%
\spic{sample_150mm.pdf}{Dies ist die Bildunterschrift. Die Verweise auf die Teilbereiche des Bildes fügt man einfach in Textform hinzu. (a) Ist zum Beispiel die Isometrie, (b) der Ausschnitt des Unterzuges usw.}{\label{testBild}}%
%
Der Befehl spicV lässt eine variable Breite des eingefügten Bildes zu. Die Breite in mm ist das dritte Argumen (Input) für die Funktion.%
%
Es ist wichtig, immer mittels Verweisen den Link zum Bild herzustellen, da \LaTeX{} das Bild an den Ort schiebt, wo es am besten Platz hat. Will man das nicht, kann man die Befehle \gf{spicH} oder \gf{spicvH} verwenden.%
%
\spicv{bild.jpg}{Weiteres Bild mit variabler Breite}{\label{test1}}{75} %
%
Allgemein ist es nicht möglich, dass das Bild aus dem Chapter hinaus verschoben wird. Es wird also allerhöchstens ans Ende des Chapter gestellt. Bevorzugt wird das Bild am Anfang der Folgeseite dargestellt. \gf{Test} \gf {Test2}%
%
\subsubsection{Achtung}%
%
Wenn Float"=Abbildungen (z.B. \textbackslash spic) eingefügt werden, soll der Befehl mit einem \%"=Zeichen abgeschlossen werden. Wird dies nicht gemacht, kann unter Umständen im Dokument an der Stelle des Befehl"=Aufrufs ein ungewünschter Leerschlag im Text vorkommen (meist am Anfang eines Abschnittes, was ein ungewolltes \gf{Einrücken} des Textes in dieser Zeile bewirkt).%
%
%
%
%
%
\section{Tabulatoren}%
%
Tabulatoren fügt man folgendermassen ein:%
%
\begin{tabbing}
    Links \= Mitte \= Rechts \\
    1 \> 2 \> 3 \\
\end{tabbing}
%
Für simpleres linksbündiges Tabbing wird das folgende Vorgehen empfohlen:\\ \\%
%
Links \tabto{3cm}Text, der nach 3cm vom linken Seitenrand her folgt.%
%
Tabbing funktioniert auch in Aufzählungen:%
%
\begin{itemize}%
\item Text \tabto{5cm} Text nach dem Tabulator%
\item Anderer Text \tabto{5cm} Anderer Text nach dem Tabulator%
\end{itemize}%
%
\section{Fussnoten}%
%
Fussnoten\footnote{Dies ist eine Fussnote.} fügt man auf diese Weise ein (am besten ohne Abstand zum betreffenden Wort). Sie werden wie bei Word automatisch\footnote{Dies ist eine andere Fussnote.} am Ende der Seite eingefügt.%
%
%
%
%
\section{Tabellen}%
%
Tabellen werden wie in den nachfolgenden Beispielen aufgezeigt generiert. Auch auf Tabellen wird immer im Text verwiesen (vgl. \autoref{tabelle1}), da diese dort im Text angezeigt werden, wo sie am besten Platz haben.%
%
\subsubsection{Erstellung mit Excel}%
%
Alternativ kann der Code für die einzelnen Zellen (vgl. unten) auch mit Hilfe von Excel erstellt werden. Dazu wird die Tabelle zuerst in Excel erzeugt und dann als csv"=Datei gespeichert. Der Inhalt der csv"=Datei wird anschliessend in den \LaTeX{}"=Code eingefügt. Damit in der csv"=Datei das Trennzeichen \& zur Anwendung kommt, muss dieses in Windows zuerst geändert werden. Dazu geht man folgendermassen vor:%
%
\begin{itemize}%
\item Start > Systemsteuerung%
\item Region und Sprache%
\item Registerkarte Format > Weitere Einstellungen%
\item Registerkarte Zahlen > Listentrennzeichen (zu \gf{\&} ändern)%
\end{itemize}%
%
%
\subsection{Spaltenformat}%
%
Es gibt folgende Optionen für das Spaltenformat:%
%
\begin{table}[H]
\begin{center}
\begin{tabularx}{\textwidth}{lX}
|    &    vertikale Linie über die gesamte Tabellenhöhe\\
||   &    vertikale Doppellinie über die gesamte Tabellenhöhe\\
l    &    linksbündige Einträge\\
c    &    zentrierte Einträge\\
r    &    rechtsbündige Einträge\\
p\{br\}    &    Der Spalteninhalt wird im Blocksatz eingefügt. Die Breite der Spalte wird durch br angegeben.\\
L\{br\}    &    linksbündig mit Breitenangabe\\
C\{br\}    &    zentriert mit Breitenangabe\\
R\{br\}    &    rechtsbündig mit Breitenangabe\\
X    & tabularx bietet den Spaltentyp X der iterativ eine automatische Spaltenverbreiterung durchführt, bis die gewünschte Gesamtbreite der Tabelle erreicht wird. Leider ist dieser Prozess langsam. Selbst auf neuen Computern verlangsamt sich der Kompilier-Vorgang bei einigen Tabellen erheblich.
\end{tabularx}
\end{center}
\end{table}
%
%
%
%
Für \gf{l}, \gf{c} und \gf{r} gilt:\\%
Jede Spalte wird so breit, wie der breiteste Zelleneintrag der Spalte. Es ist kein Zeilenumbruch möglich!%
%
%
%
%
\newpage%
\subsection{Einfachste Tabelle}%
%
\begin{table}[h]
\begin{center}
\begin{tabular}{rlr}
   Position & Beschreibung & Anzahl \\
          1 & Lenkrad      &      1 \\
          2 & Reifen       &      4 \\
          3 & Motor        &      1 \\
\end{tabular}
\end{center}
\stabcaption{Tabelle ohne horizontale Linien, Breite wird automatisch nach Zelleninhalt gewählt}         
\label{tab:1}
\end{table}
%
%
%
\subsection{Tabelle mit \textbackslash multicolumn und \textbackslash multirow}%
%
\begin{table}[h]
\begin{center}
\begin{tabular}{rlr}
   \hline
   Position & Beschreibung & Anzahl \\
   \hline
          1 & Lenkrad      &      \multirow{3}{*}{2} \\ %Sternchen bedeutet automatische Breite
          2 & Reifen\\
          3 & Motor\\
  \hline
  \multicolumn{3}{r}{Gesamtanzahl: 6} \\
  \hline
\end{tabular}
\end{center}
\stabcaption{Tabelle mit horizontalen Linien, eine Spalte \gf{verschmolzen}, Breite automatisch nach Inhalt}         
\label{tab:2}
\end{table}
%
%
%
%
\subsection{Tabelle Spaltentyp p}%
%
\begin{table}[h]
\begin{center}
\begin{tabular}{lrp{5cm}}
   \hline
   \textbf{Text} & \textbf{Text} & \textbf{Text} \\
   \hline
   linksbündig & 
   rechtsbündig & 
   Typ p (hier 5 cm breit) Hier kommt ziemlich viel Text hinein, der mehrere Zeilen beanspruchen kann. Hier kommt ziemlich viel Text hinein, der mehrere Zeilen beanspruchen kann. Hier kommt ziemlich viel Text hinein, der mehrere Zeilen beanspruchen kann.  \\
  \hline
\end{tabular}
\end{center}
\stabcaption{mit dem Spaltentyp p kann die geforderte Spaltenbreite vorgegeben werden, es entsteht ein Blocksatz, Zeilenumbrüche werden automatisch gemacht}         
\label{tab:3}
\end{table}
%
%
%
\newpage%
\subsection{Tabelle mit definierten Spaltentypen}%
%
\begin{table}[h]
\begin{center}
\begin{tabular}{C{3cm}R{4cm}L{3.5cm}}
   \hline
   \textbf{Text} & \textbf{Text} & \textbf{Text} \\
   \hline
   Diese Tabelle benutzt eigenen Spaltentyp \emph{C}.
   & Diese Tabelle benutzt eigenen Spaltentyp \emph{R}. 
   & Diese Tabelle benutzt eigenen Spaltentyp \emph{L}. \\
   zentriert mit angegebener Breite & 
   rechtsbündig mit angegebener Breite & 
   linksbündig mit angegebener Breite\\
  \hline
\end{tabular}
\end{center}
\stabcaption{Tabelle mit definierten Spaltentypen \gf{C\{$\dots$ mm\}}, \gf{R\{$\dots$ mm\}} und \gf{L\{$\dots$ mm\}}}         
\label{tab:4}
\end{table}
%
%
%
%
\subsection{Gesamte Tabelle mit Breitenangabe}%
%
Hier wird die Breitenangabe \gf{\textbackslash textwith} gewählt, somit ist die Tabelle genau so breit wie der Text auf der restlichen Seite.%
%
\begin{table}[H]
\begin{center}
\begin{tabularx}{\textwidth}{C{30mm}R{30mm}X}
\hline
\textbf{Spalte 1} & \multicolumn{2}{c}{\textbf{Spalte 2}}\\
\cline{2-3}
 & hallo 1 & hallo 2 \\
\hline
diese erste Spalte ist 3~cm breit, und zentriert & 
diese Spalte ist rechtsbündig und ebenso breit & 
Diese Spalte benutzt den Spaltentyp X \\
\hline
\end{tabularx}
\end{center}
\stabcaption{Beispieltabelle mit den \gf{columntype} C\{30mm\} für die erste, R\{30mm\} für die zweite und X für die dritte Spalte}         
\label{tab_test}
\end{table}
%
%
 \subsection{Tabelle mit Graufärbung}%
%
\begin{table}[H]
\begin{center}
\begin{tabular}{lll>{\columncolor{grey3}}l}
\rowcolor{grey2}
Nr. & Text   & Anzahl & Titel \\
0   & hallo   & 0     & 0   \\
\rowcolor{grey3}
1   & hallo & 0     & 1   \\
2   & \cellcolor{grey2}hallo  & 0     & 2   \\
31  & hallo & 3 & 7
\end{tabular}
\end{center}
\stabcaption{Beispieltabelle mit gefärbten Zellen}
\end{table}
%
%
%
%
\newpage%
\subsection{Weitere Beispieltabellen}%
%
\begin{table}[H]
\begin{center}
\begin{tabular}{l|p{3cm}|r}\hline\hline
links & p--Spalte & rechts\\\hline
A & jetzt hat diese Spalte eine fixe Breite und ein \gf{\textbackslash newline} \newline
sorgt für eine neue Zeile in der Spalte & B \\\cline
{2-2}
1 & 2 & 3\\\hline
\end{tabular}
\end{center}
\stabcaption{Beispieltabelle mit zusammengefassten Zellen in der zweiten Spalte} 
\end{table}
%
%
%
\begin{table}[H]
\begin{center} 
\begin{tabularx}{9cm}{|X|X|X|}
\hline
In dieser Tabelle & hat jede Zelle genau die gleich Breite & nämlich gerade 
3cm \\
\hline
Und wie man dabei leicht erkennen kann & reicht diese Breite nicht bei allen
 & Spalten aus um den gesamten Text darzustellen. \\
\hline
\end{tabularx}
\end{center}
%
\stabcaption{Beispieltabelle mit der Umgebung \gf{tabularx} zur variablen Definition der Spaltenbreite.}
%
\end{table}
%
%
%
\subsection{Kombination verschiedener Tabellenarten}%
%
%
\begin{table}[H]
\begin{center}
\begin{tabularx}{150mm}{ l c c c X }
\hline 
\textbf{Parameter mit Input-Bereich} & \textbf{Bez.} & \textbf{Min} & \textbf{Max} & \textbf{Kommentar} \\
\hline 
Mittlere Betondruckfestigkeit & $f_{cc}$ & $33\,\textrm{MPa}$ & $58\,\textrm{MPa}$ & gemäss \cite{siadoku0192,eurocode2}  \\ 
Elastizitätsmodul Beton & $E_{c,28}$ & $19\,\textrm{GPa}$ & $46\,\textrm{GPa}$ & resultierend aus Beschränkung der Betondruckfestigkeit in \newline \cite{siadoku0192,eurocode2}\\ 
Betonspannung& $\vec{\sigma_{c}}$ & $0$ & $-0.4f_{cm}$ & Beschränkung gemäss \cite{siadoku0192}, \newline Input entspricht Werte-Vektor für zeitlichen Verlauf  \\ 
Zeitvektor zu $\vec{\sigma_{c}}$ & $\vec{t_{\sigma_{c}}}$ & 0 & $t_{end}$ & Zeitargumente zum zeitlichen Verlauf von $\vec{\sigma_{c}}$ (Werte-Vektor) \newline Erster Wert $\le t_0$ \\
Querschnitt & $A_{c}$ &  &  & $h_0$ beschränkt \\ 
Umfang & $u$ & &  & $h_0$ beschränkt \\ 
Bezogene Bauteildicke & $h_0$ \footnotesize{$=f(A_c,u)$} & $100\,\textrm{mm}$ & $600\,\textrm{mm}$ & gemäss \cite{siadoku0192,eurocode2} \\  
Rohdichte Leichtbeton& $\rho_{LC}$ & - & - & keine Angabe zum zulässigen \newline  Bereich an Inputparametern  \\ 
Umgebungsfeuchte & $\vec{RH}$ & 5\,\% & 95\,\% & Beschränkung gemäss \cite{siadoku0192,eurocode2}, \newline Werte-Vektor \\ 
Zeitvektor zu $\vec{RH}$ & $\vec{t_{RH}}$ & 0 & $t_{end}$ & Zeitargumente zum zeitlichen Verlauf von $\vec{RH}$ (Werte-Vektor) \newline Erster Wert $\le t_0$ \\
Temperatur beim Erhärten & $T$ & $0\,^{\circ}\mathrm{C}$ & $40\,^{\circ}\mathrm{C}$ & gemäss \cite{siadoku0192,eurocode2}  \\
\noalign{\vskip 3mm}    
\hline
\textbf{Parameter mit Argumenten} & \textbf{Bez.} & \multicolumn{2}{c}{\textbf{Argumente}} & \textbf{Kommentar} \\
\hline
Faktor Erhärtungsgeschw.& $\alpha$    & 
\multicolumn{2}{l}{
\begin{tabular}[t]{ R{4mm} l }
-1 & langsam \\
0 & normal \\
1 & schnell \\
\end{tabular}
} & 
\begin{tabular}[t]{l }
\hspace{-3mm} \footnotesize{CEM 42.5R, CEM 52.5N, CEM 52.5R} \\
\hspace{-3mm} \footnotesize{CEM 32.5R, CEM 42.5N} \\
\hspace{-3mm} \footnotesize{CEM 32.5N} \\
\hspace{-3mm} gemäss \cite{siadoku0192,eurocode2}
\end{tabular}  \\ 
Koeffizient Leichtbeton & $koef_{LC}$ & 
\multicolumn{2}{l}{
\begin{tabular}[t]{ R{4mm} l }
1 & true \\
0 & false \\
\end{tabular}
} & Kennzeichnung Vorliegen von LC \\ 
\noalign{\vskip 3mm}  
\hline
\textbf{Zeitparameter} &\textbf{Bez.} & \multicolumn{2}{l}{} & \textbf{Kommentar} \\
\hline
Endwert Zeit & $t_{end}$ & & & Ende der Auswertung \\
Zeitschritt & $\Delta t$ & & & Zeitschritte sind über die ganze Zeitspanne [$t=0$\,;\,$t_{end}$] konstant \\
Zeitpunkt Einsetzen Schwinden & $t_s$ & & & Ende der Nachbehandlung, \newline muss ein Element des Vektors [0\,:\,$\Delta t$\,:\,$t_{end}$] sein\\
\hline
\end{tabularx}
\end{center} 
\stabcaption{Inputparameter Grundfunktion Verformungen mit Minimum und Maximum}
\label{tabelle1}
\end{table}
%
%
%
%
%
\subsection{Gedrehte Tabelle} \label{refGedrehteTab}%
%
Vgl. Folgeseite%
\newpage%
%
\begin{sidewaystable}
\setlength{\tabcolsep}{4.0pt}
\begin{table} [H]
 \begin{center}
\begin{tabularx}{150mm}{ l c c c X }
\hline 
\textbf{Parameter mit Input-Bereich} & \textbf{Bez.} & \textbf{Min} & \textbf{Max} & \textbf{Kommentar} \\
\hline 
Mittlere Betondruckfestigkeit & $f_{cc}$ & $33\,\textrm{MPa}$ & $58\,\textrm{MPa}$ & gemäss \cite{siadoku0192,eurocode2}  \\ 
Elastizitätsmodul Beton & $E_{c,28}$ & $19\,\textrm{GPa}$ & $46\,\textrm{GPa}$ & resultierend aus Beschränkung der Betondruckfestigkeit in \newline \cite{siadoku0192,eurocode2}\\ 
Betonspannung& $\vec{\sigma_{c}}$ & $0$ & $-0.4f_{cm}$ & Beschränkung gemäss \cite{siadoku0192}, \newline Input entspricht Werte-Vektor für zeitlichen Verlauf  \\ 
Zeitvektor zu $\vec{\sigma_{c}}$ & $\vec{t_{\sigma_{c}}}$ & 0 & $t_{end}$ & Zeitargumente zum zeitlichen Verlauf von $\vec{\sigma_{c}}$ (Werte-Vektor) \newline Erster Wert $\le t_0$ \\
Querschnitt & $A_{c}$ &  &  & $h_0$ beschränkt \\ 
Umfang & $u$ & &  & $h_0$ beschränkt \\ 
Bezogene Bauteildicke & $h_0$ \footnotesize{$=f(A_c,u)$} & $100\,\textrm{mm}$ & $600\,\textrm{mm}$ & gemäss \cite{siadoku0192,eurocode2} \\  
Rohdichte Leichtbeton& $\rho_{LC}$ & - & - & keine Angabe zum zulässigen \newline  Bereich an Inputparametern  \\ 
Umgebungsfeuchte & $\vec{RH}$ & 5\,\% & 95\,\% & Beschränkung gemäss \cite{siadoku0192,eurocode2}, \newline Werte-Vektor \\ 
Zeitvektor zu $\vec{RH}$ & $\vec{t_{RH}}$ & 0 & $t_{end}$ & Zeitargumente zum zeitlichen Verlauf von $\vec{RH}$ (Werte-Vektor) \newline Erster Wert $\le t_0$ \\
Temperatur beim Erhärten & $T$ & $0\,^{\circ}\mathrm{C}$ & $40\,^{\circ}\mathrm{C}$ & gemäss \cite{siadoku0192,eurocode2}  \\
\hline
\end{tabularx}
\end{center}    
\stabcaption{Inputparameter Grundfunktion Verformungen mit Minimum und Maximum \blindtext}     
\label{tabelle1b}
\end{table}
\end{sidewaystable}
%
%
%
%
%
%
%
%
%
%
\section{Quellcode}%
%
%
%
Nachfolgend wird eine Matlab"=Codestelle in den \LaTeX{}"=Bericht eingefügt.%
%
\textbf{Quelle:} Beispiel (33) aus \href{http://public.rz.fh-wolfenbuettel.de/~coriand/mlab/BeispielTestAufgaben.pdf}{Link}%
%
\lstsethslu{Matlab}{Captiontext ist hier}{co_intervall}
\begin{lstlisting}
clear all; clc;
%
% Nullstellenberechnung mittels Bisektion
%
f=@(x) exp(-x)-4*x; %anonymous function
a=-2; aa=a;
b=2; bb=b;
tol=10^(-10);
anz_iter=0;
disp('Funktion muss in [a,b] stetig sein!');
if (f(a)*f(b)<0)
    while (abs(b-a)>tol)    (*@\label{co_while}@*)
        m=(b+a)/2;
        anz_iter=anz_iter+1;
        if (f(m)*f(b)<0)
            a=m;
        else
            b=m;
        end
    end
    xs=m; ys=f(m);
    xs
    f(xs)
    anz_iter
else
    disp('Intervall falsch gewaehlt');                           
end
%----------------------------------------
% zur Probe: Funktion zeichnen auf dem Ausgangsintervall
x=aa:0.05:bb;
n=length(x);
for k=1:näääääääää
    y(k)=f(x(k));ÜÜÜÜÜÜÜÜÜ
end
plot(x,y,'b-',xs,ys,'md');
Repetition Code
clear all; clc;
%
% Nullstellenberechnung mittels Bisektion
%
f=@(x) exp(-x)-4*x; %anonymous function
a=-2; aa=a;
b=2; bb=b;
tol=10^(-10);
anz_iter=0;
disp('Funktion muss in [a,b] stetig sein!');
if (f(a)*f(b)<0)
    while (abs(b-a)>tol)    (*@\label{co_while}@*)
        m=(b+a)/2;
        anz_iter=anz_iter+1;
        if (f(m)*f(b)<0)
            a=m;
        else
            b=m;
        end
    end
    xs=m; ys=f(m);
    xs
    f(xs)
    anz_iter
else
    disp('Intervall falsch gewaehlt');                           
end
%----------------------------------------
% zur Probe: Funktion zeichnen auf dem Ausgangsintervall
x=aa:0.05:bb;
n=length(x);
for k=1:n
    y(k)=f(x(k));
end
plot(x,y,'b-',xs,ys,'md');
Repetition Code
clear all; clc;
%
% Nullstellenberechnung mittels Bisektion
%
f=@(x) exp(-x)-4*x; %anonymous function
a=-2; aa=a;
b=2; bb=b;
tol=10^(-10);
anz_iter=0;
disp('Funktion muss in [a,b] stetig sein!');
if (f(a)*f(b)<0)
    while (abs(b-a)>tol)  
        m=(b+a)/2;
        anz_iter=anz_iter+1;
        if (f(m)*f(b)<0)
            a=m;
        else
            b=m;
        end
    end
    xs=m; ys=f(m);
    xs
    f(xs)
    anz_iter
else
    disp('Intervall falsch gewaehlt');                           
end
%----------------------------------------
% zur Probe: Funktion zeichnen auf dem Ausgangsintervall
x=aa:0.05:bb;
n=length(x);
for k=1:n
    y(k)=f(x(k));
end
plot(x,y,'b-',xs,ys,'md');
\end{lstlisting}

\autoref{co_while} (Verweis auf Zeile) ist zentral, ab hier beginnt die while-Schleife. \gf{\autoref{co_intervall}} (Verweis auf ganzen Codeblock) ist das Label für den gesamten Code-Block. \\%
%
Der Code kann auch aus dem PDF herauskopiert werden, ohne dass die Zeilennummerierung oder die Seitenzahlen/Kopfzeilen mitkopiert werden.%
%
%
%
%
%
%
\section{Wasserzeichen}%
%
Der Befehl \gf{\textbackslash watermark} im Preamble des Hauptfiles sorgt für ein Wasserzeichen mit dem Text \gf{Entwurf} auf dem Titelblatt (Option \gf{truefirstpage}) oder auf allen Seiten (Option \gf{trueall}).
%
%
%
%
%
%
%
%
\section{Ein paar Zeichen in \LaTeX}%
%
\begin{itemize}%
\item Prozent \% %
\item Anführungs- und Schlusszeichen "` "' oder \glqq \grqq. Schneller geht es mit dem hsluBT-Command \gf{Wort}%
\item Schlusszeichen vor und nach dem Wort: "'Wort"' %
\item Griechische Buchstaben: $\alpha$, $\beta$, $\gamma$ , $\Delta$, $\epsilon$, usw.%
\begin{itemize}%
\item Spezieller Command für aufrechtes My: $\upmu$ (anstelle von $\mu$)%
\item Spezieller Command für sauberes Phi: $\varphi$ (anstelle von $\phi$)%
\end{itemize}%
\item Wortteile die nicht getrennt werden dürfen: \mbox{et al.} %
\item Wörter mit Bindestrichen verbinden, die eine Trennung des Wortes zulassen: Spannungs"=Dehnungs"=Verhalten. %
\item Werte mit Einheiten: $f_{cm} = 20.0 \, \textrm{MPa}$ (man beachte den Abstand zwischen Zahl und Einheit). Schneller geht es mit diesem hsluBT-Command: $f_{cm} = 20.0 \ee{MPa}$%
\item Kleiner Abstand bei Multiplikation zweier Variablen: $a = b\, c$ (sieht besser aus als $a = b c$) %
\item Grad Celsius kann innerhalb von Formeln ebenfalls mit einem hsluBT-Command erzeugt werden: $20.1 \gc$%
%
\end{itemize}%
%
%
%
%
%
%
\section{ToDo-Liste erstellen}%
%
Das todonotes Paket ermöglicht es farbige (gut sichtbare) ToDo-Einträge in das Dokument einzufügen. Am Ende des Dokuments kann mithilfe von \textbackslash listoftodos eine Liste mit allen noch offenen ToDo's erstellt werden (siehe allerletzte Seite dieses Dokuments). Um die Verweis-Linien korrekt darzustellen sind (mindestens) zwei Kompiliervorgänge nötig. %
%
\subsection{Beispiele von ToDo-Einträgen}%
%
\subsubsection{einfach}%
%
Hier steht ein Beispieltext mit \todo{Ergänzung} einem gut sichtbaren Hinweisfenster, was noch zu tun ist. %
%
\subsubsection{Farben geändert}%
%
Hier steht wieder ein bisschen Text. \todo[linecolor=green, backgroundcolor=blue,bordercolor=red]{anything but default} Hier steht ein bisschen Text. Hier steht ein bisschen Text. Hier steht ein bisschen Text. \blindtext \todo[color=yellow]{yellow note}%
%
\subsubsection{ohne Linie}%
%
Ein todonotes-Beispiel ohne Linie. Diese Notiz erscheint nicht in der \textbackslash listoftodos \todo[noline,nolist]{Eine Notiz ohne Linie}.%
%
\newpage%
\subsubsection{Platzhalter für ein noch fehlendes Bild}%
%
Wenn ein Platzhalter für ein noch nicht vorhandenes Bild benötigt wird, kann der Code \gf{\textbackslash missingfigure[figwidth=XX, figheight=XX]{XX}} eingefügt werden. Hier ein Beispiel mit Textbreite und 6\,cm Höhe.\\%
\missingfigure[figwidth=\linewidth,figheight=6cm]{Hier muss noch ein Bild hin}%
%
\subsection{final option}%
Wenn im Hauptdokument (hier \gf{Vorlagen\_{}XX.tex}) in Zeile 1 bei \textbackslash documentclass[a4paper, fleqn, german]{book} zusätzlich die Option final (also kurz vor dem Abgabe-Termin) geladen wird, dann verschwinden die \textbackslash listoftodos am Ende und alle gemachten ToDo-Einträge.%
%