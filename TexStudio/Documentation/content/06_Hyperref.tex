% Hyperref
%
Ein mit der vorliegenden Vorlage erstelltes \LaTeX "=Dokument enthält zahlreiche Hyperref"=Links. Dies bedeutet, dass sämtliche Verweise im PDF auch direkt als Link fungieren. Klickt man den Verweis an, landet man auf der entsprechenden Seite. Hyperref"=Links funktionieren für:%
%
\begin{itemize}%
\item Bilder%
\item Tabellen%
\item Überschriften%
\item Gleichungen%
\item Codeblöcke%
\item Quellenverweise%
\end{itemize}%
%
\section{Backref} \label{refHyperefSec}%
%
In den Quellenverweisen ist jeweils die Informaiton enthalten, auf welchen Seiten auf diese Quelle verwiesen wird (Zitiert auf Seite ...). Auch diese Verweise funktionieren wiederum als Links im PDF.%
%
\section{Autoref}%
%
Bei Verweisen, welche mit der Funktion \as{\textbackslash autoref}, bzw, \as{\textbackslash aref} oder \as{\textbackslash autoeqref} formuliert werden, wird beispielsweise der Begriff \as{Bild} automatisch dazugeschrieben. Dabei fungiert zudem im PDF nicht nur die Zahl als Link, sondern auch der ganze dazugehörige Begriff (z.B. \autoref{testBild} anstatt nur \ref{testBild}).%
%
das funktioniert natürlich auch mit:\\ \\%
\autoref{testBild}\\%
\autoref{tab:1}\\%
\autoref{refHyperref}\\%
\autoref{refHyperefSec}\\%
\autoref{refGedrehteTab}\\ \\%
%
Im Fall von Anhängen wird der \as{\textbackslash aref\{\}}-Befehl (\textit{a} für appendix oder Anhang) benötigt:\\ \\%
%
\aref{refChapterAnhang}\\%
\aref{refSectionAnhang}\\ \\%
%
Bei Formeln kann der Befehl \as{\textbackslash autoeqref} angewendet werden: \\ \\%
%
\autoeqref{Formellabel11} \hspace{10mm}%
%
\subsubsection{Tipp}%
%
\begin{itemize}%
\item Labels mit einem Kürzel beginnen, das Auskunft darüber gibt, auf welche Art von Textbaustein es verweist (z.B. \as{picBeispiel} für Bild, \as{tabBeispiel} für Tabelle, \as{eqBeispiel} für Gleichung, \as{refBeispiel} für Überschriften).%
\end{itemize}%
%