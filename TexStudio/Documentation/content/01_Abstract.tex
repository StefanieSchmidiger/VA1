% Vorwort
%
With unmanned vehicles, there are always on-board and off-board components. Data transmission between those components is of vital importance. Depending on the distance between on-board and off-board components, different data transmission technologies have to be used. \\
In this project, a hardware has been designed where multiple data inputs and outputs and multiple transmitters can be connected to a serial switch. The designed hardware features an SD card with a configuration file where data routing can be configured. \\
Data from connected devices will be collected and put into a data package with header, checksum, time stamp and other information. The package is then sent out via the configured transmitter. The corresponding second serial switch hardware receives this package, extracts and checks the payload, sends it out to the corresponding device and sends an acknowledge back to the package sender. \\
When data transmission over one transmission technology fails, the configuration file lets the user select the order of back up transmitters to be used. Data priority can also be configured because reliability of data transmission is extremely important with information such as exact location of the drone but not as important with information such as state of charge of the battery. \\
The serial switch hardware designed in the scope of this project features four serial RS232 connections where input and output devices can be connected that process or generate data. There are also four RS232 connectors where transmitters can be connected to send or receive data packages. The routing between data generating devices and transmitters to use can be done in a .ini file saved on an SD card. \\
There are two SPI to UART converters that act as the interface between the four devices connected and the micro controller respectively the four transmitters and the micro controller. \\
In a first version of the project, a Teensy 3.1 development board has been used as a micro controller unit. The software was written in the Arduino IDE with the provided Arduino libraries. As the project requirements became more complex, the limit of only a serial interface available as a debugging tool became more challenging. In the end, the first version of the software ran with more than ten tasks and an overhaul of the complex structure was necessary.\\
For this reason, an adapter board has been designed so the existing hardware could be used with the more powerful Teensy 3.5. This  adapter board features a SWD hardware debugging interface that was ready to use after removing a single component on the Teensy 3.5 development board. \\
The Teensy 3.5 was then configured to run with FreeRTOS. Task scheduler and queues provided by this operating system have been used to develop software that extracts data from received packages to output them on the configured interface or generates packages from received data bytes to send them out over the configured transmitter. The concept of acknowledges has also been applied so package loss can be detected and lost packages can be resent. \\
The software concept implemented is easy to understand, maintainable and expandable. Even though the functionality of the finished project remains the same as in the first version with Teensy 3.1 and Arduino, a refactoring has been necessary. Now further improvements and extra functionalities can be implemented more easily. 