% Hardware
%
The task description for this project as seen in \todo{Aufgabenstellung} requires the following hardware changes:
%
\begin{itemize}%
    \item Optimization of size and weight %
    \item Optimization for outdoor use%
    \item Usage of more powerful processor with more memory and RNG or encryption support %
    \item SWD/JTAG debugging interface%
    \item UART hardware flow control
    \item On-board SD card (regular or micro)
\end{itemize}% \\
%
There are several options on how to proceed with the implementation. The pro and cons of these choices are listed in this chapter, followed by the chosen solution and its execution.
%
%
%
\section{Hardware Redesign Options}
The hardware Andreas Albisser designed is working as expected. The desired modifications as listed in the task description are merely optimizations. Only the replacement of the Teensy 3.1 is absolutely required for this project because software will later be written for a specific micro controller. \\
Therefore there are two options on how to proceed.\\
%
\begin{itemize}
    \item Redesign entire hardware for/with a new processor.
    \item Redesign hardware step by step, starting with just an adapter board to use existing hardware with new micro controller.
\end{itemize}
%
\subsection{Complete Hardware Redesign}
A redesign of the entire hardware requires careful component selection, adaption of the schematic and footprints and redesign of the PCB. \\
Changes for the next complete hardware redesign include:
\begin{itemize}
    \item New RS232 level to TTL level converter with more inputs to convert hardware flow control pins (CTS/RTS) as well.
    \item A way to switch between RS232 level and TTL level for all serial interfaces accessible to the user.
    \item More powerful micro controller with support for encryption.
    \item SD card slot.
    \item Hardware debugging interface.
    \item Connectors for all serial interfaces with more pins to include hardware flow control.
\end{itemize}
Implementing all these changes would require about three weeks, followed by one week of manufacturing and one week of assembly. In case of a faulty hardware, it would be extremely difficult start with the software implementation because there is no hardware to work with and no way to work standalone. In this case, producing a second version of the hardware would take up a considerable amount of time because of manufacturing time, assembly and testing.\\
There is simply not enough time to redesign the entire hardware in the scope of this project.\\
A possible project plan for this scenario can be found in \autoref{Project plan complete HW redesign}:\\
\begin{figure}[H]
    \centering
    \includegraphics[width=1\textheight, angle=90, origin=c]
    {ProjectPlan_CompleteHwRedesign.pdf}
    \caption{Possible project plan of a complete hardware redesign}
    \label{Project plan complete HW redesign}
\end{figure}
%
\subsection{Adapter Board}
The most profound hardware change is the replacement of the Teensy 3.1.\\
First, a new development board or micro controller has to be selected that supports hardware debugging and meets the requirements for data encryption. \\
After selecting a replacement for the Teensy 3.1, the fastest way to get started with software development for the new micro controller is by designing an adapter board for the Teensy 3.1 footprint. \\
The Teensy 3.1 ships with headers that can be soldered onto the development board. Andreas Albisser designed the interface for the Teensy with female header pins so the development board could simply be plugged into the header and exchanged if needed. This makes it easy to design an adapter print with male headers that match the footprint of the previously used Teensy 3.1. \\
This solution has been chosen in the scope of this work to ensure that the end result of this project would provide solid ground work for further development.
%
%
%
\section{Component Evaluation}
Before starting with the design of an adapter board, a replacement for the Teensy 3.1 development board has to be chosen. \\
%
\subsection{Development Board Selection}
The easiest option is to select a more powerful Teensy development board that meets the requirements listed in \todo{Link zu Aufgabenstellung}. \\
Fortunately, both the Teensy 3.5 and Teensy 3.6 meet the requirements and have an on-board SD card slot. A comparison between the Teensies can be found in \autoref{Teensy comparison}
%
\begin{table}[h]
    \begin{center}
        \begin{tabular}{l|L{4cm}L{4cm}L{4cm}}
             & \textbf{Teensy 3.1} & \textbf{Teensy 3.5} & \textbf{Teensy 3.6} \\
             \hline
             %-------------------------------------------------------------------------------------
            \textbf{Processor} & MK20DX256 \newline 32 bit ARM \newline Cortex-M4 \newline 72 MHz & 
            MK64FX512VMD12 \newline Cortex-M4F \newline 120 MHz & 
            MK66FX1M0VMD18 \newline Cortex-M4F \newline 180 MHz \\
            \textbf{Flash Memory [bytes]} & 262 k & 512 k & 1024 k \\
            \textbf{RAM Memory [bytes]} & 65 k & 196 k & 256 k \\
            \textbf{EEPROM [bytes]}	 & 2048 & 4096 & 4096 \\
            \textbf{I/O} & 34, 3.3V, 5V tol & 58, 3.3V, 5V tol & 58, 3.3V, 5V tol \\
            \textbf{Analog In} & 21 & 27 & 25 \\
            \textbf{PWM} & 12 & 17 & 19 \\
            \textbf{UART,I2C,SPI} & 3 & 6 & 6 \\
            \textbf{SD Card} & no & yes & yes \\
            \textbf{Price} & \$19.80 & \$25.25 & \$29.25 \\
        \end{tabular}
    \end{center}
    \stabcaption{Teensy comparison}         
    \label{Teensy comparison}
\end{table}
%
The pins of both Teensy 3.5 and 3.6 are backwards compatible to the pin out of Teensy 3.2 which will make it easier to develop the PCB of an adapter board. \\
The Teensy 3.5 and Teensy 3.6 development board have all pins needed for SWD hardware debugging available as pads on their backside. \\
The Teensy 3.5 was chosen for this application because there is more support available for this component and a FreeRTOS that is already configured. This is not the case with the Teensy 3.6.\\
%
%
\subsection{Preparation for Hardware Debugging}
The Teensy development boards are meant for USB programming and debugging. They are equipped with a small micro controller that acts as a boot loader. The small micro controller is in control of the hardware debugging and reset pins of the main micro controller and does the programming of the main micro controller. This way, all Teensies can be used with standard Arduino libraries and programmed with the Arduino IDE. \\
The schematic of the Teensy 3.5 can be seen in \autoref{Schematic Teensy 3.5}. The MKL02Z32VFG4 acts as the boot loader and the MK64FX512 is the main micro controller.\\
\spic{Teensy35_Schematic.png}{Schematic Teensy 3.5}{\label{Schematic Teensy 3.5}}
\spic{Teensy35_PinAssignment_FrontSide.png}{Pin assignment, front side}{\label{Pin assignment, front side}}
\spic{Teensy35_PinAssignment_BackSide.png}{Pin assignment, back side}{\label{Pin assignment, back side}}
The pins available to the user are shown in \autoref{Pin assignment, front side} and \autoref{Pin assignment, back side}. \\
%
\subsubsection{Serial Wire Debug}
\spic{SWD_Pinout.png}{SWD pinout}{\label{SWD pinout}}
As can be seen in \autoref{Pin assignment, back side}, there are SWD (Serial Wire Debug) pins are available as pads on the back side of the Teensy 3.5. \\
The pinout of a SWD interface can be seen in \autoref{SWD pinout}. Only the reset, data, clock and ground pins are absolutely required to be connected.\\
Even though those SWD pins are available on the backside of both Teensy 3.5 and Teensy 3.6, they are controlled by the boot loader. \\
There are two ways to communicate to the main micro controller directly without the boot loader interfering on the hardware debugging interface:
\begin{itemize}
    \item Holding the boot loader in reset mode
    \item Removing the boot loader
\end{itemize}
\subsubsection{Resetting the Boot Loader} \label{Resetting the Boot Loader}
\spic{Pinout_Bootloader.JPG}{Pin out MKL02Z32VFG4}{\label{Pinout boot loader}}
According to the data sheet of the boot loader (see \autoref{Pinout boot loader} ), pin 15 can have one of three functions: \\
\begin{itemize}
    \item Reset
    \item GPIO input
    \item GPIO output
\end{itemize}
As default, the pin will be configured as a reset pin, but this function can be turned off by configuring to any of the other two functions in software. \\
Even though the Teensies are fully open source, the software for the boot loader is not available. The only way to find out if the reset pin is still configured as such is by pulling it low and attempting a reset. \\
Before putting the boot loader into reset mode, any other functions that it may be responsible for need to be ensured. \\
Because the internal pull ups of the boot loader are used for the reset line of the main micro controller, this reset line needs to be pulled up externally first. \\
\spic{ResettingBootLoader.jpg}{Trying to pull the boot loader into reset mode}{\label{Pulling boot loader reset low}}
A resistor can be soldered onto the Teensy directly for this purpose as seen in \autoref{Pulling boot loader reset low}. Afterwards, pin 15 of the boot loader can be pulled low. \\
Unfortunately, pin 15 seems to be configured as a GPIO pin because pulling the boot loaders reset line low does not prevent it from communicating to the Teensy 3.5. \\
\spic{Teensy35_uC_keeps_Talking_Even_with_RESET_pulled_to_3V3.png}{The boot loader keeps communicating to the Teensy}{\label{Bootloader keeps talking}}
The state of the hardware debugging pins in idle state were checked with the scope but as can be seen from \autoref{Bootloader keeps talking}, the boot loader is still in control of the debugging interface and therefore the main micro controller.\\
Instead of investigating further into this option, the second option was chosen. \\
%
\subsubsection{Removing the Boot Loader}
The MKL02Z32VFG4 has two functions:
\begin{itemize}
    \item It acts as a boot loader and controls the SWD interface to the main micro controller.
    \item Its internal pull ups are used to keep the main micro controllers reset line high.
\end{itemize}
To leave the user in full control of the SWD hardware debugging interface, the boot loader has to be removed (or silenced, as attempted in \ref{Resetting the Boot Loader}). \\
Flux gel was applied around the boot loader before heating the soldering pads up and removing the MKL02Z32VFG4. Now a pull up resister was added as seen in \autoref{Resetting boot loader}. Afterwards, the SWD debugging interface could be used. \\
%
%
%
%
%
%
%
\section{To-Do List for next version of UAV serial switch}%
%    \\b\be\beg\begi\begin\begin{\begin{i\begin{it\begin{ite\begin{itemize
List all the changes that are needed on the UAV base board\\
HW Flow control can't be done without changing HW because lines are RS232 level!