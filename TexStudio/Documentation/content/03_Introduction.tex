% 023_Introduction
%
This work is being done for Aeroscout GmbH, a company that specialized in development of drones. \\
With unmanned vehicles, there are always on-board and off-board components. Data transmission between those components is of vital importance. Depending on the distance between on-board and off-board components, different data transmission technologies have to be used. \\
So far, each device that generates or processes data was directly connected to a modem. This is fine while the distances between on-board and off-board components does not vary significantly. But as soon as reliable data transmission is required both in near field and far field, the opportunity of switching between different transmission technologies is vital. When data transmission with one modem becomes unreliable, an other transmission technology should be used to uphold exchange of essential information such as exact location of the drone.\\
The goal of this project is to provide a flexible hardware that acts as a switch between devices and modems. Data routing between all connected devices and modems should be configurable and data priority should be taken into account when transmission becomes unreliable.\\
It should be possible to transmit the same data over multiple modems to reduce the chance of data loss for vital information. At receiving side, this case should be handled so the original information can be reassembled correctly with the duplicated data received. In case of data loss or corrupted data received, a resend attempt should be started.\\
The configuration should be read from a file on an SD card. This SD card should also be used to store logging data. The system should run with Free RTOS and have a command/shell interface. When no devices are connected, the Free RTOS should go into low power mode.\\
Data loss should be handled and encryption and interleaving should be implemented for data transmission.\\
A hardware should be designed that is ready for field, with a good choice of connectors, small and light weight.\\
It was not necessary to start from scratch for this project. Andreas Albisser has already developed a hardware with four RS232 interfaces to connect different data generating and processing devices and four RS232 interfaces to connect modems. As a micro controller he used the Teensy 3.2, a small, inexpensive and yet powerful USB development board that can be used with the Arduino IDE.\\
Andreas Albisser also developed a software for the designed UAV Serial Switch base board. The software concept implemented became more complex as the requirements were expanded during development. The finished product did not fulfill all requirements of Aeroscout GmbH. Therefore this follow up project was initiated with new requirements and the hope of a better and easier expandable software as an outcome.\\
Not all requirements can possibly be implemented within one semester, but good ground work should be provided for further modifications and expansions.\\
Because encryption requires a more powerful micro controller than has been used by Andreas Albisser, some hardware modifications are required in the scope of this project. The most profund change is the micro controller and usage of Free RTOS. This will therefore be the main focus inside the project. The aim is to have a stable application with at least as many features and working configuration parameters as the old software had.\\
Some requirements demand hardware changes on the base board so an evaluation needs to be done inside this project to decide how to proceed and where to invest time.\\
A detailed task description can be found in chapter two. An overview and critical analysis of the hardware and software provided by Andreas Albisser is in chapter three. In chapter four, all hardware changes that have been done in the scope of this project are described, followed by chapter four with a description of the software developed. Chapter six is for the conclusion and lessons learned.