% 02_Kurzfassung
%
With unmanned vehicles, there are always on-board and off-board components. Data transmission between those components is of vital importance. Depending on the distance between vehicle and ground station, different data transmission technologies have to be used. \\
In a previous project, the hardware for a Serial Switch has been designed that features four RS-232 interfaces to connect data processing and generating devices and four RS-232 interfaces to connect modems for data transmission. The application running on the designed base board assembled data packages with the received data from its devices and sent those data packages out to the modems for transmission. The corresponding second Serial Switch received those data packages, checked them for validity and extracted the payload to send it out to its devices.\\
A Teensy 3.2 development board acted as the main micro controller. Teensies are small, inexpensive and powerful USB development boards for Arduino applications. The software developed was flexible and in its header files the user could configure individual baud rates for each RS-232 interface, data routing and the use of acknowledges for data packages for each modem side. The application was running with many tasks, was complex and not easy to debug because of no hardware debugging interface.\\
Then this follow up project was initiated with the aim of an application with better maintainability and expandability. The main requirement for this follow up project were the use of a more powerful micro controller with Free FROS as an operating system, the use of an SD card for a configuration file and data logging and a hardware debugging interface.\\
In the scope of this project, the Teensy 3.2 was replaced with a Teensy 3.5 development board, which featured an on-board SD card slot. The Teensy 3.5 was prepared for hardware debugging and an adapter board to use the new Teensy 3.5 with headers meant for the Teensy 3.2 was designed. This adapter board allowed the use of the same base board as was designed in the previous project.\\
For the Teensy 3.5 application, the concept with data packages is applied as well and the same configuration parameters are used. The configuration is read from an .ini file saved on the SD card.\\
The functionality of the application remains the same as in the Teensy 3.2 software but with better maintainability and an easier software concept with less tasks. Hardware debugging is now possible which is of vital importance for this application to be further expandable.\\
The Teensy 3.2 application was neither well documented nor running stable. While the Teensy 3.5 application provides the same functionalities as the previous software, all its issues are documented and possible workarounds are suggested. Data handling and data loss in case of unreliable data transmission channel is handled better and the application is running stable.