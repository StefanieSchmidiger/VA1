\documentclass[a4paper,fleqn,german]{book}
%
%---------------------------------------------------------------
%  LaTeX - Template, Grundlagen, Tipps, Vorlagen (Version 09)
%  (c) April 2016, Manuel Wipfli, Stefan Lisibach
%  Hochschule Luzern - Technik und Architektur
%  Abteilung Bautechnik
%---------------------------------------------------------------
%
%
%
%--------------------------------------------------------------------------------------------------------------------
% PREAMBLE
%--------------------------------------------------------------------------------------------------------------------

\usepackage{corefiles/hsluBTmaster13}

\hypersetup{%
pdfcreator={pdflatex},
pdfproducer={LaTeX},
pdftitle={Titel der Arbeit},              %%% Titel der Arbeit UNBEDINGT ANPASSEN!
pdfsubject={Thema},                       %%% Thema (subject) UNBEDINGT ANPASSEN!
pdfauthor={Autor Mustermann},             %%% Autor UNBEDINGT ANPASSEN!
pdfkeywords={Stichwörter},                %%% Stichwörter UNBEDINGT ANPASSEN!
bookmarksnumbered=true,                   %%% Nummerierte Bookmarks
bookmarksopen=true,                       %%% Bookmarks bei PDF-öffnen bereits geöffnet?
colorlinks=false,                         %%% Farbig markierte Links
plainpages=false,                         %%% zur korrekten Erstellung der Bookmarks
bookmarksopenlevel=1,												 %%% Bookmarks nur bis Hierarchiestufe Section geöffnet
pdfpagelabels,                            %%% zur korrekten Erstellung der Bookmarks
hidelinks,                                %%% Links verstecken
pdfpagelayout=TwoPageRight                %%% Voreingestellte Ansicht im PDF-Editor (z.B. Acrobat)
}%

\graphicspath{{pictures/}}                %%% Pfad, wo die Bilder abgelegt werden
\bibliography{corefiles/literatur}        %%% Datei für die Literaturquellen

\watermark{truefirstpage} % Wasserzeichen "Entwurf": (trueall, truefirstpage,false)


%--------------------------------------------------------------------------------------------------------------------
% DOKUMENT
%--------------------------------------------------------------------------------------------------------------------

%- - - - - - - - - - - - - - - - - - - - - - - - - - - - - - - - - - - - - - - - - - - - - - - - - - - - - - - - - - 
\begin{document}													% Nicht editieren!
\lsstyle                               % Ab hier Zeichenabstand +10 (Nicht editieren!)
\fontsize{10.5}{13.7}\selectfont       % In nachfolgenden Seiten Font 10.5pt (Nicht editieren!)
\pagenumbering{alph}                   % Nötig für Richtigkeit von backref-Verweisen (Nicht editieren!)
%- - - - - - - - - - - - - - - - - - - - - - - - - - - - - - - - - - - - - - - - - - - - - - - - - - - - - - - - - - 


%--------------------------------------------------------------------------------------------------------------------
% TITELBLATT, VERSIONSTABELLE UND SELBSTSTÄNDIGKEITSERKLÄRUNG
%--------------------------------------------------------------------------------------------------------------------

\prestuffmastershort                                 % Funkt. die TB, Versionstab. und Selbstst.-Erkl. generiert
{                                                    % 1. Input kann auch selber noch angepasst werden
\huge\textbf{\LaTeX}\\                               %%% Titel der Arbeit 1. Zeile 
\vspace{2mm}
\huge\textbf{HSLU Bautechnik Master}\\     			     %%% Titel der Arbeit 2. Zeile
\vspace{8mm}
\Large\textbf{Template, Grundlagen, Tipps, Vorlagen} %%% Untertitel der Arbeit
}
{Vertiefungsmodul I}                         	       %%% Art der Arbeit
{Stefan Lisibach, Manuel Wipfli}          	      		 	%%% Autor
{Karl Mustermann}                          	       		%%% Advisor
{Berta Beispiel}                                   		%%% Experte
{Horw}                                          		 	%%% Ort
{2015}                                         		 		%%% Jahr XXXX
{XX.XX.}                                       		 		%%% Tag und Monat der Selbstständigkeitserklärung XX.XX.
{Version 0 & Vorabzug & 01.09.14 & Franz Muster}    	%%% Änderungsverzeichnis, für neue linie \\




%--------------------------------------------------------------------------------------------------------------------
% VORWORT
%--------------------------------------------------------------------------------------------------------------------

\vorwort%
{true} % ist Vorwort vorhanden? (true,false)
{% Vorwort
%
Hier wird der Lauftext des Vorworts eingefügt.%} % Text des Vorworts
{Horw, im Februar 2015} % Ort, Datum
{Franz Muster} % Verfasser des Vorworts


%--------------------------------------------------------------------------------------------------------------------
% ZUSAMMENFASSUNG
%--------------------------------------------------------------------------------------------------------------------

\zusammenfassung%
{truedeutsch} % Ist Abstract vorhanden?(truebothsamepage, truebothseparatepages, truedeutsch, false)
{% 02_Kurzfassung
%
Hier wird der gesamte Text der Kurzfassung eingefügt.%} % Text der Kurzfassung in Deutsch
{} % Text der Kurzfassung in Englisch


%--------------------------------------------------------------------------------------------------------------------
% INHALTSVERZEICHNIS (automatisch generiert)
%--------------------------------------------------------------------------------------------------------------------

%- - - - - - - - - - - - - - - - - - - - - - - - - - - - - - - - - - - - - - - - - - - - - - - - - - - - - - - - - - 
% outsource_TOC
%- - - - - - - - - - - - - - - - - - - - - - - - - - - - - - - - - - - - - - - - - - - - -
\clearpage
\frontmatter      % Beginn der römischen Seitenzahlen
\pagestyle{fancy}       % Für nachfolgende Seiten
\markboth{Inhaltsverzeichnis}{}
\chapter*{Table of Contents} 
\pdfbookmark[1]{Inhaltsverzeichnis}{Inhaltsverzeichnis}
\vspace{-12.8mm}
\makeatletter
\@starttoc{toc}    % Generieren des Inhaltsverzeichnisses
\makeatother
%- - - - - - - - - - - - - - - - - - - - - - - - - - - - - - - - - - - - - - - - - - - - - 	% Nicht editieren!
%- - - - - - - - - - - - - - - - - - - - - - - - - - - - - - - - - - - - - - - - - - - - - - - - - - - - - - - - - - 


%--------------------------------------------------------------------------------------------------------------------
% BEGINN DER NUMMERIERTEN KAPITEL
%--------------------------------------------------------------------------------------------------------------------

%- - - - - - - - - - - - - - - - - - - - - - - - - - - - - - - - - - - - - - - - - - - - - - - - - - - - - - - - - - 
\mainmatter%         	% Nicht editieren!
\pagestyle{fancy}%	% Nicht editieren!
%- - - - - - - - - - - - - - - - - - - - - - - - - - - - - - - - - - - - - - - - - - - - - - - - - - - - - - - - - - 


\chapter{Einleitung}%
% Einleitung
%
\section{Zu diesem Dokument}
%
Das vorliegende Dokument dient als Nachschlagewerk und als Vorlage beim Erstellen von Dokumenten mit \LaTeX . Die etwas seltsame Form mit vielen Platzhalterkapiteln hat seinen Grund: Werden die einzelnen .tex"=Dateien im Ordner \gf{content/} entfernt (oder erstetzt), kann dieses (dann fast vollkommen leere) Dokument für jegliche Berichte von Vertiefungsarbeiten und Master"=Thesen der HSLU/MSE direkt als Vorlage übernommen werden. \\\\%
Die folgenden Abschnitte sollten weitgehend alle benötigten Informationen enthalten, die zur Herstellung der einzelnen Bausteine eines \LaTeX "=Berichtes benötigt werden. Ein Ausdruck dieser Vorlage ist insofern nur beschränkt dienlich, da bei sämtlichen Erklärungen davon ausgegangen wird, dass der Leser neben der fertigen PDF"=Datei auch den tex"=Quellcode vor sich hat. So kann zum einen direkt verifiziert werden, welcher Code welchen Output generiert und ferner können auf diese Weise gewünschte Codestellen für den Eigengebrauch direkt übernommen werden.%
%
\section{Wann \LaTeX ? Wann nicht?}%
%
Die Anwendung von \LaTeX zahlt sich vor allem im Fall von umfangreichen wissenschaftlichen Arbeiten aus. Die Vorteile, welche bei deren Erstellung zum Zuge kommen, werden im Kapitel \ref{seVorteileLatex} näher vertieft.%
%
\spicvH{Latex_effort}{\LaTeX und MS Word: Qualitative Darstellung des Aufwandes in Funktion der Dokument"=Komplexität}{\label{picAufwand}}{50}%
%
Bild \ref{picAufwand} zeigt in qualitativem Sinne, dass \LaTeX{} den WYSIWYG"=Editoren (what you see is what you get) wie beispielsweise \gf{MS Word} nur ab einem gewissen Mass an Dokument"=Komplexität überlegen ist. Dieses Mass an Komplexität ist in den MSE"=Dokumenten definitiv erreicht und die bisherigen Nutzer"=Feedbacks sind ausnahmslos sehr positiv.%
%
\section{Vorteile von \LaTeX} \label{seVorteileLatex}%
%
Die hier aufgeführten Vorteile beziehen sich auf alle Features die \LaTeX{} standardmässig bietet, oder die durch das Anwenden von Zusatzfunktionen durch das hsluBTmaster-Stylefile ermöglicht werden.%
%
\begin{itemize}
\item Weltweiter Standard in der Wissenschaft und im Engineering%
\item \LaTeX{} ist Freeware%
\item \LaTeX{} wird permanent auf der ganzen Welt weiterentwickelt%
\item Läuft extrem stabil, egal wie gross die Dokumente sind%
\item \LaTeX -Dateien sind extrem klein%
\item Layout eines Dokuments wird nicht beeinflusst, wenn die Datei mit einer neueren Version von \LaTeX{} editiert wird%
\item Gefahr, dass man durch Unachtsamkeit etwas im Layout ungewollt verändert, ist fast ausgeschlossen%
\item Diverse verschiedene Distriubtionen und Editoren für \LaTeX{} vorhanden, alle funktionieren aber genau gleich und generieren bei gleichem Input den selben Output%
\item Kann bei Bildern mit fast allen Bildformaten umgehen, darunter JPG, PNG, PS, EPS, \linebreak \underline{und vor allem PDF}, Vektorgrafiken sind selbstredend auch nach dem Kompilieren noch in Vektorform gespeichert%
\item  Anpassen von bereits in das Dokument eingefügte Bilddateien funktioniert sehr effizient.%
\item Aus dem \LaTeX -Dokument generierte PDF's haben eine geringe Dateigrösse und besitzen viele Zusatz-Features:%
%
\begin{itemize}
\item Bookmarks für das gesamte Dokument, die beim öffnen der Datei bereits sichtbar sind%
\item Im PDF anklickbare Link-Funktionen (sind im hsluBTmaster-Style-File bereits so eingestellt). %
%
\begin{itemize}
\item Jeder Eintrag im Inhaltsverzeichnis führt zum entsprechenden Kapitel%
\item Querverweise auf Gleichungen, Bilder und Tabellen führen zu der entsprechenden Abbildung%
\item Literaturverweise führen zur entsprechenden Stelle im Literaturverzeichnis%
\item Rückverweise, bei jeder Literaturstelle ist verzeichnet, auf welcher Seite sie im Bericht verzeichnet ist. Der Verweis funktioniert wiederum als anklickbarer Link.%
\end{itemize}
\item Hinzufügen von Metadaten wie Titel, Thema, Autor, Stichworte, etc.%
\item Doppelseiten-Ansicht beim öffnen, Titelblatt aber separat (wenn so gewünscht)%
\end{itemize} 
%
\item Eingeben von Formeln verlangt keine Mausklicks für Sonderzeichen, Brüche, Operatoren etc., und ist somit einiges schneller%
\item Es können mehrere Teildateien desselben Berichts gleichzeitig geöffnet sein (schnelles Arbeiten alleine, Kollaboration möglich)%
\item Sehr leichter Umgang beim Zitieren aus Quellen, alle Dokumente können auf dasselbe persönliche Bibliography-File zugreifen%
\item Sehr gute Kompatibilität mit fast allen Literaturverwaltungsprogrammen wie Zotero, Jabref, Citavi, etc. Drag-and-Drop-Features bei Zotero%
\item Layout wird mit dem hsluBTmaster-Stylefile automatisch erstellt, ohne dass man sich darum kümmert, somit ist auch das Layout von Autor zu Autor identisch%
\item Kann ganze PDF-Seiten (oder gar mehrseitige PDF-Dokumente) in das Dokument einfügen oder anhängen, z.B. bei Handgeschriebenen Seiten in einer Statik o. ä.%
\item Extrem aktive Community im Internet bei Fragen%
\item Viele Templates im Internet vorhanden, z.B. für Briefe, etc.%
\item Funktion "'Beamer"' für Powerpoint-Präsentationen%
\item Möglichkeiten unbegrenzt, beliebig ausbaufähig, programmieren eigener Commands möglich%
\end{itemize}
%
%
%
%
%
%
\section{Installation}
%
\subsection{Vorgehen}
%
Zur Bearbeitung von \LaTeX -Dateien muss eine \LaTeX "=Distribution und ein Editor heruntergeladen werden. Die folgenden beiden Programme haben sich als geeignet erwiesen:%
%
\begin{itemize}
\item Distribution \gf{TexLive}, \\ https://www.tug.org/texlive/acquire-netinstall.html (install-tl-windows.exe)%
\item Editor \gf{Texmaker}, \\ http://www.xm1math.net/texmaker/download.html%
\end{itemize}
%
Beide Programme sind auch für Mac erhältlich. Bei der Installation muss unbedingt zuerst die Distribution komplett installiert werden, bevor der Editor dazu installiert wird, so dass dieser die Distribution bereits vorfindet und sich damit verknüpfen kann. Im Normalfall sollte dies problemlos funktionieren.\\%
Tritt beim Kompilieren dennoch eine Fehlermeldung auf, welche \gf{latex -interaction=nonstopmode \%.tex} beinhaltet, hat TexMaker die Distribution nicht gefunden. In diesem Fall müssen unter \gf{Optionen} > \gf{Texmaker konfigurieren} > \gf{Befehle} die Pfade zu den Programmdateien der Distribution manuell eingegeben werden.%
%
\subsection{To do's nach der Installation}%
%
Nach der Installation sollte das Rechtschreibewörterbuch auf Deutsch geändert werden. \linebreak \gf{Optionen}  > \gf{Texmaker konfigurieren} > \gf{Editor} > \gf{Rechtschreibewörterbuch}. Am schnellsten geht es, wenn der text \gf{en\_{}GB.dic} per Tastatureingabe in \gf{de\_{}DE.dic} geändert wird. TexMaker findet im momentanen Verzeichnis die entsprechende Datei. \\\\%
%
Des weiteren können unter \gf{Benutzer/in} > \gf{Wortvervollständigung anpassen} zusätzliche Vervollständigungen aktiviert werden, so dass man nicht ständig die gesamten Commands tippen muss. Dies macht das Arbeiten um einiges bequemer. Vorgeschlagen werden an dieser Stelle Einträge zu den folgenden Funktionen:%
%
\begin{samepage}
\begin{itemize}
\item \textbackslash cite\{o\} \tabto{5cm} Zitieren%
\item \textbackslash ee\{o\} \tabto{5cm} Einheit in Formeln%
\item \textbackslash as\{o\} \tabto{5cm} Anführungs- und Schlusszeichen%
\item \textbackslash gf\tabto{5cm} Gänsefüsschen%
\item \textbackslash gc\tabto{5cm} Grad Celsius %
\item \textbackslash jj\tabto{5cm} Kleiner Buchstabe \gf{j} bei Formeln (Zeichenabstand richtig!)%
\item \textbackslash spic\{o\}\{o\}\{o\} \tabto{5cm} Einfügen eines Bildes (float, Breite 150mm)%
\item \textbackslash spicH\{o\}\{o\}\{o\} \tabto{5cm} Einfügen eines Bildes (here, Breite 150mm)%
\item \textbackslash spicv\{o\}\{o\}\{o\}\{o\} \tabto{5cm} Einfügen eines Bildes (float, Breite wählbar)%
\item \textbackslash spicvH\{o\}\{o\}\{o\}\{o\} \tabto{5cm} Einfügen eines Bildes (here, Breite wählbar)%
\item \textbackslash textbackslash \tabto{5cm} Backslash \textbackslash%
\item \textbackslash textrm\{o\} \tabto{5cm} Für Text in Formeln%
\item \textbackslash tabto\{o\} \tabto{5cm} Tabulator%
\item \textbackslash blindtext \tabto{5cm} Erzeugt ein Fülltext %
\end{itemize}
\end{samepage}
%
Obige Funktionen sind zum Teil \LaTeX "= Standardfunktionen oder Funktionen aus dem Stylefile \gf{hsluBTmaster}.%
%%
%
\chapter{Grundlagen}%
% Grundlagen
%
\section{Zum Codefile}%
%
Ein umfangreiches \LaTeX -Dokument besteht aus Hauptfile (inoffizielle Bezeichnung), importierten Teildateien, Stylefile und der Bibliography.%
%
\subsubsection{Hauptfile (.tex)}%
%
Im Hauptfile (hier \gf{Vorlagen\_{}XX.tex}) werden alle Teildateien des Dokuments importiert und wenn nötig werden Titelblätter, Vorwörter, Verzeichnisse, etc. generiert. Grössere Textbausteine sollten in Teildateien gespeichert und über den \textbackslash import-Befehl importiert werden. So wird Übersichtlichkeit gewährleistet. (siehe Beispiel in dieser Vorlage).%
%
\subsubsection{Importierte Teildateien (.tex) (\textbackslash input"= Befehl)}%
%
Am besten wird pro Kapitel (und pro Anhangteil) eine eigene Teildatei erstellt (vgl. Ordner \gf{content/}). Auf diese Weise können mehrere Kapitel gleichzeitig geöffnet und editiert werden und das Hauptfile bleibt übersichtlich. Auch die Einträge für die Bezeichnungen können in einer Teildatei abgelegt und importiert werden (hier \gf{content/Vorlagen\_{}XX\_{}Bezeichnungen.tex}).\\%
Ferner werden im oben beschriebenen Hauptfile einige weitere nötige Bestandteile des Codes aus outsource"=Dateien eingefügt (vgl. Ordner \gf{corefiles}). Diese Dateien sollen grundsätzlich nicht editiert werden. Das Outsourcing dient wiederum hauptsächlich der Gewährleistung der Übersichtlichkeit.%
%
\subsubsection{Stylefile (.sty)}%
%
Das Stylefile (hier \gf{hsluBTmasterXX.sty} enthält alle Angaben, die \LaTeX{} braucht, um das Layout zu generieren. Darüber hinaus sind diverse sogenannte Commands hinterlegt, welche das einfache Erstellen von Titelblättern, Änderungsverzeichnissen, Selbstständigkeitserklärungen, Vorwörter, Abstracts, Bezeichnungsverzeichnissen, Literaturverzeichnissen und Bildern (und vielem mehr) ermöglichen. \\ %
Das Stylefile (hier \gf{hsluBTmasterXX.sty}) enthält am Anfang ein Änderungsverzeichnis und ein Manual, das die Anwendung der im File definierten Commands erklärt.%
%
\subsubsection{Bibliography (.bib)}%
%
Wird auf Quellen verwiesen, ist ein zusätzliches Bibliography-File nötig (hier \gf{literatur.bib}). Das Vorlagenfile enthält zahlreiche Quellen aus verschiedenen Dokument-Kategorien. %
%
\subsubsection{Weitere Files}%
%
Abgesehen von obigen Files entstehen beim Kompilieren noch andere Dateien (.bbl, .log, .blg, .toc, .out, .aux, etc.) Diese Dateien werden komplett von \LaTeX{} gehandhabt. Wenn ein Dokument dupliziert oder transportiert wird (z.B. auf einen anderen Datenträger) müssen diese Dateien nicht mitkopiert werden. Sie werden beim nächsten Kompilieren neu generiert. Es ist aber damit zu rechnen, dass in solchen Fällen zwingend mehrere Kompilier"=Durchgänge nötig sind.%
%
\newpage%
\section{Zur Anwendung des Stylefiles \gf{hsluBTmaster}}%
%
Wenn ein Bericht auf dieser Vorlage aufgebaut wird, sind die nachfolgenden Bedingungen bereits erfüllt und diesem Abschnitt muss somit keine Beachtung geschenkt werden.\\ \\%
Damit das Stylefile \gf{hsluBTmaster} angewendet werden kann, und das damit beabsichtigte Layout komplett umgesetzt werden kann, müssen auch beim Hauptfile einige Bedingungen eingehalten werden.%
%
\begin{itemize}%
\item Documentclass muss \gf{book} sein.%
\item Bei der Documentclass müssen die fakultativen Argumente [a4paper, fleqn, german] lauten%
\end{itemize}%
%
Damit das Layout wie in dieser Vorlage aussieht, müssen zudem diverse Codestellen analog zum Hauptfile dieses Dokuments eingesetzt sein. Es sind dies:%
%
\begin{itemize}%
\item Im Preamble (alles vor \gf{\textbackslash begin\{document\}}):%
\begin{itemize}%
\item \textbackslash usepackage\{corefiles/hsluBTmasterXX\} \textit{Laden des Stylefiles}%
\item \textbackslash hypersetup \textit{Ausfüllen diverser Parameter und Metadaten für das resultierende PDF}%
\item \textbackslash graphicspath\{\{pictures/\}\}   \textit{Ordner für die Bilder}  %
\item \textbackslash bibliography\{corefiles/literatur\} \textit{Datei für die Literaturverweise} %
\item \textbackslash watermark\{truefirstpage\} \textit{Optionales Wasserzeichen \gf{Entwurf}}%
\end{itemize}%
\item Nach \gf{\textbackslash begin\{document\}}:%
\begin{itemize}%
\item \gf{\textbackslash lsstyle}  \textit{(regelt den Zeichenabstand zwischen den Buchstaben)}%
\item \gf{\textbackslash fontsize\{10.5\}\{13.7\} \textbackslash selectfont} \textit{(regelt die Fontgrösse / Zeilenabstand)}%
\item \gf{\textbackslash pagenumbering\{alph\}} \textit{(nur sofern unnummerierte Seiten vor dem Titelblatt benötigt werden, sorgt für korrekte Backref"=Verweise für alle unnummerierten Seiten)}%
\end{itemize}%
%
\item Bei Inhaltsverzeichnis:%
\begin{itemize}%
\item \gf{\textbackslash input\{corefiles/outsource\_{}TOC\}} \textit{(generiert das Inhaltsverzeichnis)}%
\end{itemize}%
\item Vor Beginn der eigentlichen Kapitel:%
\begin{itemize}%
\item \gf{\textbackslash mainmatter} \textit{(Beginn der normalen Seitennummerierung)}%
\item \gf{\textbackslash pagestyle\{fancy\}} \textit{(regelt Seitenzahlen und Kapitelangaben im Header)}%
\end{itemize}%
%
\item Vor Beginn des Anhangs:%
\begin{itemize}%
\item \gf{\textbackslash input\{corefiles/outsource\_{}Appendix\}} \textit{(ändert Kapitelnummerierung)}%
\end{itemize}%
%
\item Nach Ende des Anhangs:%
\begin{itemize}%
\item \gf{\textbackslash input\{corefiles/outsource\_{}endAppendix\}} \textit{(ändert Kapitelnummerierung zurück)}%
\end{itemize}%
\end{itemize}%
%
Alle diese Codestellen sollten nicht editiert werden.%
%
\subsection{Zur Funktion \gf{anhangstuff}}%
%
Die einzelnen Teile des Anhangs werden im zweiten Argument der Funktion  \gf{\textbackslash anhangstuff } eingefügt. Dabei ist \gf{\textbackslash chapter} wie gewohnt die höchste hierarchische Überschriftstufe (generiert Titel mit der alphabetischen Nummerierung A, B, C, etc.). Alle darunterliegenden Kapitel werden wie gewohnt mit  \gf{\textbackslash section},  \gf{\textbackslash subsection} und  \gf{\textbackslash subsubsection} eingefügt.%
%
\section{Nützliche Funktionen von TexMaker}%
%
\subsubsection{Blauer Pfeil (Ausführen) in "'Tools Toolbar"'}%
%
Kompiliert die Datei. Im Dropdown-Menü kann gewählt werden, wie kompiliert wird. %
\begin{itemize}%
\item Beim normalen Arbeiten: 1.  PDFLaTex 2. PDF anzeigen (wird in den Standardeinstellungen ausgelöst, wenn man "'Schnelles Übersetzen"' im Dropdown-Menü wählt. Damit das Inhaltsverzeichnis aktualisiert wird, sind in der Regel zwei solche Durchgänge nötig.%
\item Mit Bibliography: 1. PDFLaTex 2. BibLaTex 3. PDFLaTex 4. PDFLaTex 5. PDF anzeigen (kann wie im nächsten Punkt beschrieben dem Dropdown-Menü-Punkt "'Schnelles Übersetzen"' zugewiesen werden).  Grundsätzlich ist dieser Vorgang nur ganz am Schluss nötig. Beim normalen Arbeiten spielt es keine Rolle, wenn die Bibliography noch nicht auf dem aktuellsten Stand ist.%
\end{itemize}%
%
\subsubsection{Blauer Pfeil (Ansehen) in "'Tools Toolbar"'}%
%
Zeigt das zum aktuellen Dokument gehörende PDF an, ohne dass ein Kompiliervorgang ausgelöst wird. Die Schaltfläche kann auch dazu benutzt werden, um nach Absetzen des Cursors an einer beliebigen Stelle im Code zur korrespondierenden Stelle im PDF zu gelangen.%
%
\subsubsection{Optionen > Texmaker konfigurieren > Schnelles übersetzen}%
%
Auf diese Weise kann man beeinflussen, was passiert, wenn man den blauen Pfeil zum kompilieren drückt und \gf{schnelles Übersetzen} gewählt ist.%
%
\subsubsection{Optionen > Aktuelle Datei zur Master-Datei erklären}%
%
Die Datei, die gerade aktiv ist, wird zur Master-Datei. Wann auch immer man von diesem Zeitpunkt an kompiliert, wird immer das Masterfile zum als PDF generiert, egal welche tex-Datei gerade aktiv (sichtbar) ist.%
%
\subsubsection{Befehl-Vervollständigung}%
%
Tippt man einen Befehl, werden von TexMaker Vervollständigungs-Vorschläge gemacht. Mit den Pfeiltasten kann der entsprechende Befehl gewählt und mit Enter eingefügt werden. Diese Vorschläge werden auch beim Zitieren über den \textbackslash{}cite-Befehl für die Literaturquellen gemacht, sofern diese bereits im Bibliography-File hinterlegt sind.%
%
\subsubsection{Ctrl-Klick in der PDF Ansicht}%
%
Wird in der PDF-Ansicht bei gleichzeitig gedrückter Ctrl-Taste auf eine bestimmte Stelle im Dokument geklickt, gelangt man im Editor-Fenster an die entsprechende Stelle im Code.%
%
\subsubsection{Suchfunktion}%
%
Gesucht wird im Editor wie in allen anderen Programmen mit Ctrl+F.%
%
%
\subsubsection{Message/Log}%
%
Messages und Log wird unterhalb des Editorfensters angezeigt. Dort werden alle Warnungen und Fehler, die beim Kompilieren aufgetreten sind aufgeführt. Diesen Warnungen sollte in jedem Fall Beachtung geschenkt werden. Nicht alle Warnungen sind aber zwingend gravierend.%
%
\subsubsection{Struktur-Fenster}%
%
Im Strukturfenster links sind alle Kapitel und Unterkapitel im aktuell geöffneten Tex-File aufgeführt. Sind der Datei Unterdateien mit dem Befehl \_{}include hinzugefügt, taucht im Strukturfenster ein Link auf, der direkt zu dieser Datei führt.%
%
\section{Zu beachten beim Arbeiten mit \LaTeX}%
%
\begin{itemize}%
\item[--] Namen von Dateien, Bilddateien, Labels und Literaturverweisen sollten keine Umlaute und Leerschläge enthalten. Dies kann bei einigen Compilern zu Problemen führen.%
\end{itemize}%%
%
\chapter{Vorlagen für die Erstellung des Berichts}%
% Vorlagen
%
Dieses Kapitel enthält zahlreiche Vorlagen, die beim Erstellen eines \LaTeX "= Dokuments von grossem Nutzen sein können. Grundsätzlich können die hier enthaltenen Codestellen stets kopiert und entsprechend angepasst werden. Bei Fragestellungen, welche über die hier aufgeführten Punkte hinausgehen, ist das Internet zu konsultieren.%
%
\section{Leerzeilen und Absätze}%
%
Eine neue Zeile beginnt man mit dem nachfolgenden Befehl (vgl. Code) \\
Eine Leerzeile fügt man so ein: (vgl. Code) \\ \\
Wenn man nach einem Lauftext im Code eine Leerzeile einfügt, erstellt dies nur eine neue Zeile. (vgl. Code)

Wenn danach ein section, subsection oder subsubsection-Befehl kommt (wie nach diesem Abschnitt) hat die Code-Leerzeile keinen negativen Einfluss auf das Layout. (vgl. Code) 

\subsubsection{Subsubsection}

Des Weiteren haben auch mehrere aufeinanderfolgende Leerzeilen im Code keinen (zusätzlichen) Einfluss auf den Output.






Wie hier zu sehen ist.
%
\subsubsection{Wichtig!}
%
Am besten wird aber wie in diesem Codefile vorgezeigt auf das Einfügen von Zeilenumbrüchen durch Leerzeilen im Code verzichtet. Leerzeilen, welche den Code übersichtlicher gestalten sollen, werden vorteilhaft mit einem \% "=Zeichen (auskommentieren) ausgefüllt, damit sie der Compiler ignoriert und somit keine unerwünschten Nebeneffekte auftreten.
%
\subsubsection{Achtung}%
%
Es wird empfohlen, am Ende jedes Abschnittes und jedes Befehl"=Aufrufs im Code ein Auskommentier"=Zeichen (\%) zu setzen. (vgl. Code)%
%
Mit dieser Massnahme wird erreicht, dass der Compiler den \gf{Enterschlag} am Ende der Codezeile ignoriert. Dieser Enterschlag wird vom Compiler als \gf{Leerschlag} interpretiert. In 99.9\% hat dieser Leerschlag keinen Layout"=bestimmenden Einfluss. In sehr seltenen Fällen (wenn bei einem Abschnitt im PDF die letzte Linie vor dem Umbruch gerade voll wird) kann eine komplett leere Linie entstehen. Beim Aufruf von Befehlen (z.B. Einfügen eines Bildes mit \gf{\textbackslash spic}) kann u.U. am Anfang eines Absatzes ein Leerschlag resultieren, was wie ein (ungewünschtes) \gf{Einrücken} des Abschnitts aussieht. 
%
%
%
%
%
\section{Formeln}%
%
Nachfolgend sind verschiedene Arten beschrieben, wie eine Formel integriert werden kann.\\
Es ist sehr wichtig, dass man bei den Enter"=Schlägen im Code das untenstehend angewendete Vorgehen genau nachahmt, so dass die Abstände zwischen Text und Formeln stimmen. (allenfalls Leerbereiche mit \%-Zeichen einfügen, so dass die Zeile komplett auskommentiert wird).%
%
Eine Formel%
%
\begin{equation}
     \epsilon_{c\sigma}(t)=\int_0^t J(t,\tau) d\sigma_c(\tau) \label{Formellabel1}
\end{equation}
%
% Wenn man Leerzeilen einfügt und diese mit Kommentaren füllt, hat dies keinen Einfluss auf das Layout.
%
Zwei Formeln mit separater Nummerierung%
%
\begin{align}
 &f_1(x) = (x+a)(x+b) \label{Formellabel2} \\
 &f_{221}(x)= x^2 + (a+b)x + ab \label{Formellabel3}
\end{align}
%
Zwei Formeln mit separater Nummerierung und Ausrichtung beim Gleichheitszeichen%
%
\begin{align}
 f_1(x) &= (x+a)(x+b) \label{Formellabel4} \\
 f_{221}(x)&= x^2 + (a+b)x + ab \label{Formellabel5}
\end{align}
%
Zwei Formeln mit einer Nummerierung%
%
\begin{equation}
\begin{gathered}
\begin{split}
 &t_T = \sum\limits^{n}_{i=1} \Delta t_i \exp{\left(13.65-\frac{4000}{273+ T(\Delta t_i )}\right)} \\
 &f_{221}(x)= x^2 + (a+b)x + ab 
\end{split}
\end{gathered}
\label{Formellabel6}
\end{equation}
%
Zwei Formeln mit Subequation, a und b%
%
\begin{subequations} 
\begin{align} 
\renewcommand\theequation{\theparentequation.\alph{equation}} 
&f_1(x) = (x+a)(x+b)  \label{Formellabel8}\\
&f_{221}(x)= x^2 + (a+b)x + ab \label{Formellabel9}
\end{align}
\end{subequations} 
%
Mehrere Gleichungen pro Zeile mit wählbarer Position der hinteren Gleichungen und einer Nummer pro Zeile%
%
\begin{alignat}{2} % Zahl entspricht Anzahl der Spalten
&y = x^2 + bx + c     \hspace{29mm} && f(x) = x^2 + 2xy + y^2+2xy   \label{Formellabel10}\\
&y = ax^2 + bx + c + d              &&  \left(\frac{35}{f_{cc}}\right)^{0.2}     \label{Formellabel11}\\
&y = ax^2 + c                       && \theta = \frac{1}{2} \, \arccot \left( \frac{\epsilon_y - \epsilon_x}{\gamma_{xy}} \right)   \label{Formellabel11b}
\end{alignat}
%
Mehrere Gleichungen pro Zeile mit nicht wählbarer Position der hinteren Gleichungen und einer Nummer pro Zeile%
%
\begin{align}
&f_1(x) = (x+a)(x+b)                   &  &w =z  \label{Formellabel12} \\            
&f_{221}(x)= x^2 + (a+b)x + ab         &  &3w=\frac{1}{2}z \label{Formellabel13}      
\end{align}
%
Mehrere Gleichungen pro Zeile mit nicht wählbarer Position der hinteren Gleichungen und einer Nummer pro Zeile, inklusive einer Zeile ohne Nummer%
%
\begin{align}
&f_1(x) = (x+a)(x+b)           &  &w =z2x=-y  \label{Formellabel15} \\            
&f_{221}(x)= x^2 + (a+b)x + ab         &  &3w=\frac{1}{2}z2x=-y  \notag\\
&-4 + 5x=2+y   &  &w+2=-1+w2x=-y   \label{Formellabel16}         
\end{align}
%
Mehrere Zeilen mit mehreren Gleichungen mit nur einer Formelnummer%
%
\begin{equation}
\begin{aligned}
&f_1(x) = (x+a)(x+b)    \hspace{45mm}       &  &w =z  \\  
&2x=-y+\alpha_H                         &  &3w=\frac{1}{2}z 
\end{aligned}
\label{Formellabel17}
\end{equation}
%
Formeln im Lauftext werden folgendermassen geschrieben: (vgl. Code) $f_1(x) = (x+a)(x+b)$. Es empfiehlt sich, auch alle Variablen, die man erwähnt, auf diese Weise im Text einzufügen, zum Beispiel $A_c$ oder $E_{c,28}$. Auf diese Weise wird gewährleistet, dass die Formatierung in jedem Fall korrekt ist. Brüche im Text stellt man entweder so: $\sfrac{b+c}{\alpha}$, oder so $\frac{b+c}{\alpha}$ dar.%
%
\section{Bild einfügen (und referenzieren)}%
%
Ein Bild fügt man mit den dafür definierten Befehlen aus dem Style-File ein. Darauf verweisen kann man mit dem folgenden Befehl, der hier auf das \autoref{testBild} weiterleitet. Bei Gleichungen wird folgender Befehl angewendet: \eqref{Formellabel16}. Die mit diesem Befehl eingefügten Bilder haben die Breite 150mm. (für den Bild-Einfüge-Code bitte Code konsultieren)\\%
%
\spic{sample_150mm.pdf}{Dies ist die Bildunterschrift. Die Verweise auf die Teilbereiche des Bildes fügt man einfach in Textform hinzu. (a) Ist zum Beispiel die Isometrie, (b) der Ausschnitt des Unterzuges usw.}{\label{testBild}}%
%
Der Befehl spicV lässt eine variable Breite des eingefügten Bildes zu. Die Breite in mm ist das dritte Argumen (Input) für die Funktion.%
%
Es ist wichtig, immer mittels Verweisen den Link zum Bild herzustellen, da \LaTeX{} das Bild an den Ort schiebt, wo es am besten Platz hat. Will man das nicht, kann man die Befehle \gf{spicH} oder \gf{spicvH} verwenden.%
%
\spicv{bild.jpg}{Weiteres Bild mit variabler Breite}{\label{test1}}{75} %
%
Allgemein ist es nicht möglich, dass das Bild aus dem Chapter hinaus verschoben wird. Es wird also allerhöchstens ans Ende des Chapter gestellt. Bevorzugt wird das Bild am Anfang der Folgeseite dargestellt. \gf{Test} \gf {Test2}%
%
\subsubsection{Achtung}%
%
Wenn Float"=Abbildungen (z.B. \textbackslash spic) eingefügt werden, soll der Befehl mit einem \%"=Zeichen abgeschlossen werden. Wird dies nicht gemacht, kann unter Umständen im Dokument an der Stelle des Befehl"=Aufrufs ein ungewünschter Leerschlag im Text vorkommen (meist am Anfang eines Abschnittes, was ein ungewolltes \gf{Einrücken} des Textes in dieser Zeile bewirkt).%
%
%
%
%
%
\section{Tabulatoren}%
%
Tabulatoren fügt man folgendermassen ein:%
%
\begin{tabbing}
    Links \= Mitte \= Rechts \\
    1 \> 2 \> 3 \\
\end{tabbing}
%
Für simpleres linksbündiges Tabbing wird das folgende Vorgehen empfohlen:\\ \\%
%
Links \tabto{3cm}Text, der nach 3cm vom linken Seitenrand her folgt.%
%
Tabbing funktioniert auch in Aufzählungen:%
%
\begin{itemize}%
\item Text \tabto{5cm} Text nach dem Tabulator%
\item Anderer Text \tabto{5cm} Anderer Text nach dem Tabulator%
\end{itemize}%
%
\section{Fussnoten}%
%
Fussnoten\footnote{Dies ist eine Fussnote.} fügt man auf diese Weise ein (am besten ohne Abstand zum betreffenden Wort). Sie werden wie bei Word automatisch\footnote{Dies ist eine andere Fussnote.} am Ende der Seite eingefügt.%
%
%
%
%
\section{Tabellen}%
%
Tabellen werden wie in den nachfolgenden Beispielen aufgezeigt generiert. Auch auf Tabellen wird immer im Text verwiesen (vgl. \autoref{tabelle1}), da diese dort im Text angezeigt werden, wo sie am besten Platz haben.%
%
\subsubsection{Erstellung mit Excel}%
%
Alternativ kann der Code für die einzelnen Zellen (vgl. unten) auch mit Hilfe von Excel erstellt werden. Dazu wird die Tabelle zuerst in Excel erzeugt und dann als csv"=Datei gespeichert. Der Inhalt der csv"=Datei wird anschliessend in den \LaTeX{}"=Code eingefügt. Damit in der csv"=Datei das Trennzeichen \& zur Anwendung kommt, muss dieses in Windows zuerst geändert werden. Dazu geht man folgendermassen vor:%
%
\begin{itemize}%
\item Start > Systemsteuerung%
\item Region und Sprache%
\item Registerkarte Format > Weitere Einstellungen%
\item Registerkarte Zahlen > Listentrennzeichen (zu \gf{\&} ändern)%
\end{itemize}%
%
%
\subsection{Spaltenformat}%
%
Es gibt folgende Optionen für das Spaltenformat:%
%
\begin{table}[H]
\begin{center}
\begin{tabularx}{\textwidth}{lX}
|    &    vertikale Linie über die gesamte Tabellenhöhe\\
||   &    vertikale Doppellinie über die gesamte Tabellenhöhe\\
l    &    linksbündige Einträge\\
c    &    zentrierte Einträge\\
r    &    rechtsbündige Einträge\\
p\{br\}    &    Der Spalteninhalt wird im Blocksatz eingefügt. Die Breite der Spalte wird durch br angegeben.\\
L\{br\}    &    linksbündig mit Breitenangabe\\
C\{br\}    &    zentriert mit Breitenangabe\\
R\{br\}    &    rechtsbündig mit Breitenangabe\\
X    & tabularx bietet den Spaltentyp X der iterativ eine automatische Spaltenverbreiterung durchführt, bis die gewünschte Gesamtbreite der Tabelle erreicht wird. Leider ist dieser Prozess langsam. Selbst auf neuen Computern verlangsamt sich der Kompilier-Vorgang bei einigen Tabellen erheblich.
\end{tabularx}
\end{center}
\end{table}
%
%
%
%
Für \gf{l}, \gf{c} und \gf{r} gilt:\\%
Jede Spalte wird so breit, wie der breiteste Zelleneintrag der Spalte. Es ist kein Zeilenumbruch möglich!%
%
%
%
%
\newpage%
\subsection{Einfachste Tabelle}%
%
\begin{table}[h]
\begin{center}
\begin{tabular}{rlr}
   Position & Beschreibung & Anzahl \\
          1 & Lenkrad      &      1 \\
          2 & Reifen       &      4 \\
          3 & Motor        &      1 \\
\end{tabular}
\end{center}
\stabcaption{Tabelle ohne horizontale Linien, Breite wird automatisch nach Zelleninhalt gewählt}         
\label{tab:1}
\end{table}
%
%
%
\subsection{Tabelle mit \textbackslash multicolumn und \textbackslash multirow}%
%
\begin{table}[h]
\begin{center}
\begin{tabular}{rlr}
   \hline
   Position & Beschreibung & Anzahl \\
   \hline
          1 & Lenkrad      &      \multirow{3}{*}{2} \\ %Sternchen bedeutet automatische Breite
          2 & Reifen\\
          3 & Motor\\
  \hline
  \multicolumn{3}{r}{Gesamtanzahl: 6} \\
  \hline
\end{tabular}
\end{center}
\stabcaption{Tabelle mit horizontalen Linien, eine Spalte \gf{verschmolzen}, Breite automatisch nach Inhalt}         
\label{tab:2}
\end{table}
%
%
%
%
\subsection{Tabelle Spaltentyp p}%
%
\begin{table}[h]
\begin{center}
\begin{tabular}{lrp{5cm}}
   \hline
   \textbf{Text} & \textbf{Text} & \textbf{Text} \\
   \hline
   linksbündig & 
   rechtsbündig & 
   Typ p (hier 5 cm breit) Hier kommt ziemlich viel Text hinein, der mehrere Zeilen beanspruchen kann. Hier kommt ziemlich viel Text hinein, der mehrere Zeilen beanspruchen kann. Hier kommt ziemlich viel Text hinein, der mehrere Zeilen beanspruchen kann.  \\
  \hline
\end{tabular}
\end{center}
\stabcaption{mit dem Spaltentyp p kann die geforderte Spaltenbreite vorgegeben werden, es entsteht ein Blocksatz, Zeilenumbrüche werden automatisch gemacht}         
\label{tab:3}
\end{table}
%
%
%
\newpage%
\subsection{Tabelle mit definierten Spaltentypen}%
%
\begin{table}[h]
\begin{center}
\begin{tabular}{C{3cm}R{4cm}L{3.5cm}}
   \hline
   \textbf{Text} & \textbf{Text} & \textbf{Text} \\
   \hline
   Diese Tabelle benutzt eigenen Spaltentyp \emph{C}.
   & Diese Tabelle benutzt eigenen Spaltentyp \emph{R}. 
   & Diese Tabelle benutzt eigenen Spaltentyp \emph{L}. \\
   zentriert mit angegebener Breite & 
   rechtsbündig mit angegebener Breite & 
   linksbündig mit angegebener Breite\\
  \hline
\end{tabular}
\end{center}
\stabcaption{Tabelle mit definierten Spaltentypen \gf{C\{$\dots$ mm\}}, \gf{R\{$\dots$ mm\}} und \gf{L\{$\dots$ mm\}}}         
\label{tab:4}
\end{table}
%
%
%
%
\subsection{Gesamte Tabelle mit Breitenangabe}%
%
Hier wird die Breitenangabe \gf{\textbackslash textwith} gewählt, somit ist die Tabelle genau so breit wie der Text auf der restlichen Seite.%
%
\begin{table}[H]
\begin{center}
\begin{tabularx}{\textwidth}{C{30mm}R{30mm}X}
\hline
\textbf{Spalte 1} & \multicolumn{2}{c}{\textbf{Spalte 2}}\\
\cline{2-3}
 & hallo 1 & hallo 2 \\
\hline
diese erste Spalte ist 3~cm breit, und zentriert & 
diese Spalte ist rechtsbündig und ebenso breit & 
Diese Spalte benutzt den Spaltentyp X \\
\hline
\end{tabularx}
\end{center}
\stabcaption{Beispieltabelle mit den \gf{columntype} C\{30mm\} für die erste, R\{30mm\} für die zweite und X für die dritte Spalte}         
\label{tab_test}
\end{table}
%
%
 \subsection{Tabelle mit Graufärbung}%
%
\begin{table}[H]
\begin{center}
\begin{tabular}{lll>{\columncolor{grey3}}l}
\rowcolor{grey2}
Nr. & Text   & Anzahl & Titel \\
0   & hallo   & 0     & 0   \\
\rowcolor{grey3}
1   & hallo & 0     & 1   \\
2   & \cellcolor{grey2}hallo  & 0     & 2   \\
31  & hallo & 3 & 7
\end{tabular}
\end{center}
\stabcaption{Beispieltabelle mit gefärbten Zellen}
\end{table}
%
%
%
%
\newpage%
\subsection{Weitere Beispieltabellen}%
%
\begin{table}[H]
\begin{center}
\begin{tabular}{l|p{3cm}|r}\hline\hline
links & p--Spalte & rechts\\\hline
A & jetzt hat diese Spalte eine fixe Breite und ein \gf{\textbackslash newline} \newline
sorgt für eine neue Zeile in der Spalte & B \\\cline
{2-2}
1 & 2 & 3\\\hline
\end{tabular}
\end{center}
\stabcaption{Beispieltabelle mit zusammengefassten Zellen in der zweiten Spalte} 
\end{table}
%
%
%
\begin{table}[H]
\begin{center} 
\begin{tabularx}{9cm}{|X|X|X|}
\hline
In dieser Tabelle & hat jede Zelle genau die gleich Breite & nämlich gerade 
3cm \\
\hline
Und wie man dabei leicht erkennen kann & reicht diese Breite nicht bei allen
 & Spalten aus um den gesamten Text darzustellen. \\
\hline
\end{tabularx}
\end{center}
%
\stabcaption{Beispieltabelle mit der Umgebung \gf{tabularx} zur variablen Definition der Spaltenbreite.}
%
\end{table}
%
%
%
\subsection{Kombination verschiedener Tabellenarten}%
%
%
\begin{table}[H]
\begin{center}
\begin{tabularx}{150mm}{ l c c c X }
\hline 
\textbf{Parameter mit Input-Bereich} & \textbf{Bez.} & \textbf{Min} & \textbf{Max} & \textbf{Kommentar} \\
\hline 
Mittlere Betondruckfestigkeit & $f_{cc}$ & $33\,\textrm{MPa}$ & $58\,\textrm{MPa}$ & gemäss \cite{siadoku0192,eurocode2}  \\ 
Elastizitätsmodul Beton & $E_{c,28}$ & $19\,\textrm{GPa}$ & $46\,\textrm{GPa}$ & resultierend aus Beschränkung der Betondruckfestigkeit in \newline \cite{siadoku0192,eurocode2}\\ 
Betonspannung& $\vec{\sigma_{c}}$ & $0$ & $-0.4f_{cm}$ & Beschränkung gemäss \cite{siadoku0192}, \newline Input entspricht Werte-Vektor für zeitlichen Verlauf  \\ 
Zeitvektor zu $\vec{\sigma_{c}}$ & $\vec{t_{\sigma_{c}}}$ & 0 & $t_{end}$ & Zeitargumente zum zeitlichen Verlauf von $\vec{\sigma_{c}}$ (Werte-Vektor) \newline Erster Wert $\le t_0$ \\
Querschnitt & $A_{c}$ &  &  & $h_0$ beschränkt \\ 
Umfang & $u$ & &  & $h_0$ beschränkt \\ 
Bezogene Bauteildicke & $h_0$ \footnotesize{$=f(A_c,u)$} & $100\,\textrm{mm}$ & $600\,\textrm{mm}$ & gemäss \cite{siadoku0192,eurocode2} \\  
Rohdichte Leichtbeton& $\rho_{LC}$ & - & - & keine Angabe zum zulässigen \newline  Bereich an Inputparametern  \\ 
Umgebungsfeuchte & $\vec{RH}$ & 5\,\% & 95\,\% & Beschränkung gemäss \cite{siadoku0192,eurocode2}, \newline Werte-Vektor \\ 
Zeitvektor zu $\vec{RH}$ & $\vec{t_{RH}}$ & 0 & $t_{end}$ & Zeitargumente zum zeitlichen Verlauf von $\vec{RH}$ (Werte-Vektor) \newline Erster Wert $\le t_0$ \\
Temperatur beim Erhärten & $T$ & $0\,^{\circ}\mathrm{C}$ & $40\,^{\circ}\mathrm{C}$ & gemäss \cite{siadoku0192,eurocode2}  \\
\noalign{\vskip 3mm}    
\hline
\textbf{Parameter mit Argumenten} & \textbf{Bez.} & \multicolumn{2}{c}{\textbf{Argumente}} & \textbf{Kommentar} \\
\hline
Faktor Erhärtungsgeschw.& $\alpha$    & 
\multicolumn{2}{l}{
\begin{tabular}[t]{ R{4mm} l }
-1 & langsam \\
0 & normal \\
1 & schnell \\
\end{tabular}
} & 
\begin{tabular}[t]{l }
\hspace{-3mm} \footnotesize{CEM 42.5R, CEM 52.5N, CEM 52.5R} \\
\hspace{-3mm} \footnotesize{CEM 32.5R, CEM 42.5N} \\
\hspace{-3mm} \footnotesize{CEM 32.5N} \\
\hspace{-3mm} gemäss \cite{siadoku0192,eurocode2}
\end{tabular}  \\ 
Koeffizient Leichtbeton & $koef_{LC}$ & 
\multicolumn{2}{l}{
\begin{tabular}[t]{ R{4mm} l }
1 & true \\
0 & false \\
\end{tabular}
} & Kennzeichnung Vorliegen von LC \\ 
\noalign{\vskip 3mm}  
\hline
\textbf{Zeitparameter} &\textbf{Bez.} & \multicolumn{2}{l}{} & \textbf{Kommentar} \\
\hline
Endwert Zeit & $t_{end}$ & & & Ende der Auswertung \\
Zeitschritt & $\Delta t$ & & & Zeitschritte sind über die ganze Zeitspanne [$t=0$\,;\,$t_{end}$] konstant \\
Zeitpunkt Einsetzen Schwinden & $t_s$ & & & Ende der Nachbehandlung, \newline muss ein Element des Vektors [0\,:\,$\Delta t$\,:\,$t_{end}$] sein\\
\hline
\end{tabularx}
\end{center} 
\stabcaption{Inputparameter Grundfunktion Verformungen mit Minimum und Maximum}
\label{tabelle1}
\end{table}
%
%
%
%
%
\subsection{Gedrehte Tabelle} \label{refGedrehteTab}%
%
Vgl. Folgeseite%
\newpage%
%
\begin{sidewaystable}
\setlength{\tabcolsep}{4.0pt}
\begin{table} [H]
 \begin{center}
\begin{tabularx}{150mm}{ l c c c X }
\hline 
\textbf{Parameter mit Input-Bereich} & \textbf{Bez.} & \textbf{Min} & \textbf{Max} & \textbf{Kommentar} \\
\hline 
Mittlere Betondruckfestigkeit & $f_{cc}$ & $33\,\textrm{MPa}$ & $58\,\textrm{MPa}$ & gemäss \cite{siadoku0192,eurocode2}  \\ 
Elastizitätsmodul Beton & $E_{c,28}$ & $19\,\textrm{GPa}$ & $46\,\textrm{GPa}$ & resultierend aus Beschränkung der Betondruckfestigkeit in \newline \cite{siadoku0192,eurocode2}\\ 
Betonspannung& $\vec{\sigma_{c}}$ & $0$ & $-0.4f_{cm}$ & Beschränkung gemäss \cite{siadoku0192}, \newline Input entspricht Werte-Vektor für zeitlichen Verlauf  \\ 
Zeitvektor zu $\vec{\sigma_{c}}$ & $\vec{t_{\sigma_{c}}}$ & 0 & $t_{end}$ & Zeitargumente zum zeitlichen Verlauf von $\vec{\sigma_{c}}$ (Werte-Vektor) \newline Erster Wert $\le t_0$ \\
Querschnitt & $A_{c}$ &  &  & $h_0$ beschränkt \\ 
Umfang & $u$ & &  & $h_0$ beschränkt \\ 
Bezogene Bauteildicke & $h_0$ \footnotesize{$=f(A_c,u)$} & $100\,\textrm{mm}$ & $600\,\textrm{mm}$ & gemäss \cite{siadoku0192,eurocode2} \\  
Rohdichte Leichtbeton& $\rho_{LC}$ & - & - & keine Angabe zum zulässigen \newline  Bereich an Inputparametern  \\ 
Umgebungsfeuchte & $\vec{RH}$ & 5\,\% & 95\,\% & Beschränkung gemäss \cite{siadoku0192,eurocode2}, \newline Werte-Vektor \\ 
Zeitvektor zu $\vec{RH}$ & $\vec{t_{RH}}$ & 0 & $t_{end}$ & Zeitargumente zum zeitlichen Verlauf von $\vec{RH}$ (Werte-Vektor) \newline Erster Wert $\le t_0$ \\
Temperatur beim Erhärten & $T$ & $0\,^{\circ}\mathrm{C}$ & $40\,^{\circ}\mathrm{C}$ & gemäss \cite{siadoku0192,eurocode2}  \\
\hline
\end{tabularx}
\end{center}    
\stabcaption{Inputparameter Grundfunktion Verformungen mit Minimum und Maximum \blindtext}     
\label{tabelle1b}
\end{table}
\end{sidewaystable}
%
%
%
%
%
%
%
%
%
%
\section{Quellcode}%
%
%
%
Nachfolgend wird eine Matlab"=Codestelle in den \LaTeX{}"=Bericht eingefügt.%
%
\textbf{Quelle:} Beispiel (33) aus \href{http://public.rz.fh-wolfenbuettel.de/~coriand/mlab/BeispielTestAufgaben.pdf}{Link}%
%
\lstsethslu{Matlab}{Captiontext ist hier}{co_intervall}
\begin{lstlisting}
clear all; clc;
%
% Nullstellenberechnung mittels Bisektion
%
f=@(x) exp(-x)-4*x; %anonymous function
a=-2; aa=a;
b=2; bb=b;
tol=10^(-10);
anz_iter=0;
disp('Funktion muss in [a,b] stetig sein!');
if (f(a)*f(b)<0)
    while (abs(b-a)>tol)    (*@\label{co_while}@*)
        m=(b+a)/2;
        anz_iter=anz_iter+1;
        if (f(m)*f(b)<0)
            a=m;
        else
            b=m;
        end
    end
    xs=m; ys=f(m);
    xs
    f(xs)
    anz_iter
else
    disp('Intervall falsch gewaehlt');                           
end
%----------------------------------------
% zur Probe: Funktion zeichnen auf dem Ausgangsintervall
x=aa:0.05:bb;
n=length(x);
for k=1:näääääääää
    y(k)=f(x(k));ÜÜÜÜÜÜÜÜÜ
end
plot(x,y,'b-',xs,ys,'md');
Repetition Code
clear all; clc;
%
% Nullstellenberechnung mittels Bisektion
%
f=@(x) exp(-x)-4*x; %anonymous function
a=-2; aa=a;
b=2; bb=b;
tol=10^(-10);
anz_iter=0;
disp('Funktion muss in [a,b] stetig sein!');
if (f(a)*f(b)<0)
    while (abs(b-a)>tol)    (*@\label{co_while}@*)
        m=(b+a)/2;
        anz_iter=anz_iter+1;
        if (f(m)*f(b)<0)
            a=m;
        else
            b=m;
        end
    end
    xs=m; ys=f(m);
    xs
    f(xs)
    anz_iter
else
    disp('Intervall falsch gewaehlt');                           
end
%----------------------------------------
% zur Probe: Funktion zeichnen auf dem Ausgangsintervall
x=aa:0.05:bb;
n=length(x);
for k=1:n
    y(k)=f(x(k));
end
plot(x,y,'b-',xs,ys,'md');
Repetition Code
clear all; clc;
%
% Nullstellenberechnung mittels Bisektion
%
f=@(x) exp(-x)-4*x; %anonymous function
a=-2; aa=a;
b=2; bb=b;
tol=10^(-10);
anz_iter=0;
disp('Funktion muss in [a,b] stetig sein!');
if (f(a)*f(b)<0)
    while (abs(b-a)>tol)  
        m=(b+a)/2;
        anz_iter=anz_iter+1;
        if (f(m)*f(b)<0)
            a=m;
        else
            b=m;
        end
    end
    xs=m; ys=f(m);
    xs
    f(xs)
    anz_iter
else
    disp('Intervall falsch gewaehlt');                           
end
%----------------------------------------
% zur Probe: Funktion zeichnen auf dem Ausgangsintervall
x=aa:0.05:bb;
n=length(x);
for k=1:n
    y(k)=f(x(k));
end
plot(x,y,'b-',xs,ys,'md');
\end{lstlisting}

\autoref{co_while} (Verweis auf Zeile) ist zentral, ab hier beginnt die while-Schleife. \gf{\autoref{co_intervall}} (Verweis auf ganzen Codeblock) ist das Label für den gesamten Code-Block. \\%
%
Der Code kann auch aus dem PDF herauskopiert werden, ohne dass die Zeilennummerierung oder die Seitenzahlen/Kopfzeilen mitkopiert werden.%
%
%
%
%
%
%
\section{Wasserzeichen}%
%
Der Befehl \gf{\textbackslash watermark} im Preamble des Hauptfiles sorgt für ein Wasserzeichen mit dem Text \gf{Entwurf} auf dem Titelblatt (Option \gf{truefirstpage}) oder auf allen Seiten (Option \gf{trueall}).
%
%
%
%
%
%
%
%
\section{Ein paar Zeichen in \LaTeX}%
%
\begin{itemize}%
\item Prozent \% %
\item Anführungs- und Schlusszeichen "` "' oder \glqq \grqq. Schneller geht es mit dem hsluBT-Command \gf{Wort}%
\item Schlusszeichen vor und nach dem Wort: "'Wort"' %
\item Griechische Buchstaben: $\alpha$, $\beta$, $\gamma$ , $\Delta$, $\epsilon$, usw.%
\begin{itemize}%
\item Spezieller Command für aufrechtes My: $\upmu$ (anstelle von $\mu$)%
\item Spezieller Command für sauberes Phi: $\varphi$ (anstelle von $\phi$)%
\end{itemize}%
\item Wortteile die nicht getrennt werden dürfen: \mbox{et al.} %
\item Wörter mit Bindestrichen verbinden, die eine Trennung des Wortes zulassen: Spannungs"=Dehnungs"=Verhalten. %
\item Werte mit Einheiten: $f_{cm} = 20.0 \, \textrm{MPa}$ (man beachte den Abstand zwischen Zahl und Einheit). Schneller geht es mit diesem hsluBT-Command: $f_{cm} = 20.0 \ee{MPa}$%
\item Kleiner Abstand bei Multiplikation zweier Variablen: $a = b\, c$ (sieht besser aus als $a = b c$) %
\item Grad Celsius kann innerhalb von Formeln ebenfalls mit einem hsluBT-Command erzeugt werden: $20.1 \gc$%
%
\end{itemize}%
%
%
%
%
%
%
\section{ToDo-Liste erstellen}%
%
Das todonotes Paket ermöglicht es farbige (gut sichtbare) ToDo-Einträge in das Dokument einzufügen. Am Ende des Dokuments kann mithilfe von \textbackslash listoftodos eine Liste mit allen noch offenen ToDo's erstellt werden (siehe allerletzte Seite dieses Dokuments). Um die Verweis-Linien korrekt darzustellen sind (mindestens) zwei Kompiliervorgänge nötig. %
%
\subsection{Beispiele von ToDo-Einträgen}%
%
\subsubsection{einfach}%
%
Hier steht ein Beispieltext mit \todo{Ergänzung} einem gut sichtbaren Hinweisfenster, was noch zu tun ist. %
%
\subsubsection{Farben geändert}%
%
Hier steht wieder ein bisschen Text. \todo[linecolor=green, backgroundcolor=blue,bordercolor=red]{anything but default} Hier steht ein bisschen Text. Hier steht ein bisschen Text. Hier steht ein bisschen Text. \blindtext \todo[color=yellow]{yellow note}%
%
\subsubsection{ohne Linie}%
%
Ein todonotes-Beispiel ohne Linie. Diese Notiz erscheint nicht in der \textbackslash listoftodos \todo[noline,nolist]{Eine Notiz ohne Linie}.%
%
\newpage%
\subsubsection{Platzhalter für ein noch fehlendes Bild}%
%
Wenn ein Platzhalter für ein noch nicht vorhandenes Bild benötigt wird, kann der Code \gf{\textbackslash missingfigure[figwidth=XX, figheight=XX]{XX}} eingefügt werden. Hier ein Beispiel mit Textbreite und 6\,cm Höhe.\\%
\missingfigure[figwidth=\linewidth,figheight=6cm]{Hier muss noch ein Bild hin}%
%
\subsection{final option}%
Wenn im Hauptdokument (hier \gf{Vorlagen\_{}XX.tex}) in Zeile 1 bei \textbackslash documentclass[a4paper, fleqn, german]{book} zusätzlich die Option final (also kurz vor dem Abgabe-Termin) geladen wird, dann verschwinden die \textbackslash listoftodos am Ende und alle gemachten ToDo-Einträge.%
%%
%
\chapter{Hyperref} \label{refHyperref}%
% Hyperref
%
Ein mit der vorliegenden Vorlage erstelltes \LaTeX "=Dokument enthält zahlreiche Hyperref"=Links. Dies bedeutet, dass sämtliche Verweise im PDF auch direkt als Link fungieren. Klickt man den Verweis an, landet man auf der entsprechenden Seite. Hyperref"=Links funktionieren für:%
%
\begin{itemize}%
\item Bilder%
\item Tabellen%
\item Überschriften%
\item Gleichungen%
\item Codeblöcke%
\item Quellenverweise%
\end{itemize}%
%
\section{Backref} \label{refHyperefSec}%
%
In den Quellenverweisen ist jeweils die Informaiton enthalten, auf welchen Seiten auf diese Quelle verwiesen wird (Zitiert auf Seite ...). Auch diese Verweise funktionieren wiederum als Links im PDF.%
%
\section{Autoref}%
%
Bei Verweisen, welche mit der Funktion \as{\textbackslash autoref}, bzw, \as{\textbackslash aref} oder \as{\textbackslash autoeqref} formuliert werden, wird beispielsweise der Begriff \as{Bild} automatisch dazugeschrieben. Dabei fungiert zudem im PDF nicht nur die Zahl als Link, sondern auch der ganze dazugehörige Begriff (z.B. \autoref{testBild} anstatt nur \ref{testBild}).%
%
das funktioniert natürlich auch mit:\\ \\%
\autoref{testBild}\\%
\autoref{tab:1}\\%
\autoref{refHyperref}\\%
\autoref{refHyperefSec}\\%
\autoref{refGedrehteTab}\\ \\%
%
Im Fall von Anhängen wird der \as{\textbackslash aref\{\}}-Befehl (\textit{a} für appendix oder Anhang) benötigt:\\ \\%
%
\aref{refChapterAnhang}\\%
\aref{refSectionAnhang}\\ \\%
%
Bei Formeln kann der Befehl \as{\textbackslash autoeqref} angewendet werden: \\ \\%
%
\autoeqref{Formellabel11} \hspace{10mm}%
%
\subsubsection{Tipp}%
%
\begin{itemize}%
\item Labels mit einem Kürzel beginnen, das Auskunft darüber gibt, auf welche Art von Textbaustein es verweist (z.B. \as{picBeispiel} für Bild, \as{tabBeispiel} für Tabelle, \as{eqBeispiel} für Gleichung, \as{refBeispiel} für Überschriften).%
\end{itemize}%
%%
%
\chapter{Literaturverweise}%
% Literaturverweise
%
\section{Bibliography und Zotero}%
%
Die Einträge im Bibliography-File können mit Zotero erstellt werden. Wenn die entsprechende Literatur dort bereits eingetragen ist, kann sie einfach per Drag-and-Drop in das BibLaTex"=Literaturfile gezogen werden. Als Alternative kann per Rechtsklick auf die Datei über den Befehl "'ausgewählten Eintrag exportieren"' ein neues BibLaTex-File mit dem Eintrag erstellt werden. Dies funktioniert auch, wenn mehrere Dateien angewählt sind.\\%
%
Bei MSE-Berichten sind sämtliche Literaturstellen in der Zotero-Datenbank abzulegen. Zum Eintragen der benötigten Attribute (Titel, Autor etc.) kann Tabelle \ref{tabLiteraturstellen} konsultiert werden. Folgende sind Punkte zu beachten:%
%
\begin{itemize}%
\item \textbf{Bevor man bei Zotero eine Literaturstelle hinzufügt, ist zu prüfen, ob diese bereits existiert.} Allfällig bemerkte doppelte Einträge werden fusioniert.%
\item Der Name der heraufgeladenen PDF-Datei soll dem Schema \as{Jahr - Autor - Titel} folgen. Also zum Beispiel \as{2009 - Seelhofer - Ebener Spannungszustand im Betonbau.pdf}. Bei MSE"=Dokumenten schreibt man zusätzlich die das Modul dazu, also beispielsweise: \as{2013 - Stenz - VM2 - Kontinuierliche Spannungsfeldmodelle.pdf}.%
\item Beim Eintrag einer Literaturstelle in Zotero ist unter \as{Datum} immer nur das Jahr einzutragen, Ausnahme: Zeitschriftenartikel (dort wenn vorhanden den Monat auch berücksichtigen).%
\item Bei Vertiefungsmodulen ist unter \as{Art des Berichtes} der Eintrag \as{Bericht Vertiefungsmodul 2} zu machen. Der Zusatz \as{Bericht} wird im Hinblick auf die Zitierung in \LaTeX{} der Verständlichkeit halber benötigt.%
\item Beim Literaturtyp \as{Bericht} werden in Zotero \as{Seiten} (von-bis) und nicht die \as{Anzahl der Seiten} verlangt. Meistens soll im Literaturverweis aber \as{123 S.} (Seitenanzahl) und nicht \as{S. 123-127} (gewisse Seiten eines Dokuments) stehen. Die erste Darstellung kann erzwungen werden, wenn in Zotero im Feld \as{Seiten} der Eintrag \as{123 S.} und nicht nur \as{123} gemacht wird. Letzterer Eintrag würde zur meist unerwünschten Darstellung  \as{S. 123} im Literaturverzeichnis führen.%
\item Um in \LaTeX{} auf eine aus Zotero exportierte Literaturstelle zu verweisen, wird im Argument des \textbackslash cite-Befehl folgendes Muster verlangt: \as{Autor}\_ \as{1.Wort des Titels } \_ \as{Jahr} . Beispiel: Auf \as{Ebener Spannungszustand im Betonbau} von Seelhofer (2009) wird mit \as{\textbackslash cite\{seelhofer\_ ebener\_ 2009\}} zitiert.%
\item Achtung: In Zotero zusätzlich eingegebene Informationen (übrige, unbenutzte Felder) können unter Umständen auch in \LaTeX{} im Literaturverzeichnis erscheinen (z.B. wenn bei einem Buch der ISBN eingegeben wird, wird dieser am Ende des Verweises im Literaturverzeichnis aufgeführt).%
\item Die Argumente \as{@keywords} und \as{@file} in BibLaTex-Literaturdatenbanken entstehen automatisch beim Export aus Zotero und haben keinen Einfluss auf den Output im Literaturverzeichnis. Sie können also in der Datenbank belassen werden.%
\item Bei Zeitschriftenartikeln muss bei Verweisen keine Seitenangabe gemacht werden, z.B. \cite{rusch_researches_1960}. In allen anderen Fällen muss die Seitenzahl, von der die Information aus der Quelle entnommen wurde, angegeben werden, z.B. \cite[S. 34]{seelhofer_ebener_2009} mit \as{\textbackslash cite$[$S. 34$]$\{seelhofer\_ ebener\_ 2009\}}%
\end{itemize}%
%
%
%
%
%
%
\begin{sidewaystable}
\setlength{\tabcolsep}{2.5pt}
\begin{table} [H]
\begin{center}
{\fontsize{7}{9}\selectfont
\begin{tabularx}{230 mm}{ l l l | c c c c c c c c c c c c c c l  c }
\hline 
Literaturtyp&Typ Zotero&Typ \LaTeX &\multicolumn{15}{ c }{Attribute} \\
&&& {\footnotesize Titel}&Autor&Nr. Bericht&Art Bericht&Ort&Institution&Seiten&Anz. Seiten&Datum&Verlag&Name Konf.&Band&Ausgabe&Publikation\\
&&&{\tiny title } &{\tiny author } & {\tiny number } & {\tiny type } & {\tiny location } & {\tiny institution } & {\tiny pages } & {\tiny pagetotal } & {\tiny year } & {\tiny publisher } & {\tiny eventtitle }& {\tiny volume } & {\tiny issue /number } & {\tiny journaltitle } \\
\hline 
Bericht \cite{grob_ermudung_1977} &Bericht&report&X&X&Nr. 75& {\tiny Bericht} &X&X&000 S.&&Jahr&&&\\
Buch \cite{wehnert_beitrag_2006} &Buch&book&X&X&&&&&&000&Jahr&X&&&&\\
Dissertation \cite{seelhofer_ebener_2009} &Dissertation&thesis&X&X&&&X&X&&000&Jahr&&&\\
Diskussionsbericht \cite{haller_schwinden_1940}&Bericht&report&X&X&Nr. 124& {\tiny Diskussionsbericht} &X&X&000 S.&&Jahr&&&\\
Konferenz-Paper, -bericht \cite{szepe_bemessung_1956} &Konferenz-Paper&{\tiny inproceedings} &X&X&&&&&00-00&&Jahr&&X&X&\\
MSE Master-Thesis \cite{amsler_bemessung_2013} &Bericht&report&X&X&&{\tiny Master-Thesis } &X&X&000 S.&&Jahr&&&&\\
MSE Bericht VM1, VM2 \cite{amsler_verstarkung_2012} &Bericht&report&X&X&& {\tiny Bericht Vertiefungsmodul 1} &X&X&000 S.&&Jahr&&&&&\\
Norm \cite{eurocode2} \cite{_model_2010} \cite{_sia_2013} &Bericht&report&X&&&&X&X&000 S.&&Jahr&&&&&&\\
Norm Dokumentation \cite{siadoku0192} &Bericht&report&X&&&&X&X&000 S.&&Jahr&&&&&\\
Anleitung / Manual \cite{teschl_matlab_2001} &Bericht&report&X&X&& {\tiny Anleitung  (o.ä.) } &X&&000 S.&&Jahr&&&&&\\
Versuchsbericht \cite{amsler_durchstanzversuch_2013} \cite{muttoni_bemessen_1988} &Bericht&report&X&X&& {\tiny Versuchsbericht }&X&X&000 S.&&Jahr&&&&&\\
Vorlesungsskript \cite{menn_langzeit-vorgange_1977} &Manuskript&report&X&X&& {\tiny Vorlesungsskript } &X&X&000 S.&&Jahr&&&&&\\
Zeitschriftenartikel \cite{rusch_researches_1960} \cite{trost_auswirkungen_1967} &Zeitschriftenart.&article&X&X&&&&&00-00&&Monat.Jahr&&& {\tiny V. 00 oder 00 } & {\tiny No. 00 oder 00 } &X\\
\hline
\end{tabularx}
}
\end{center}   
\stabcaption{Für die Literaturverweise benötigte Informationen beim Heraufladen auf Zotero und Zitieren in \LaTeX}
\label{tabLiteraturstellen}
\end{table}
\end{sidewaystable}%



%--------------------------------------------------------------------------------------------------------------------
% ANHANG
%--------------------------------------------------------------------------------------------------------------------

%- - - - - - - - - - - - - - - - - - - - - - - - - - - - - - - - - - - - - - - - - - - - - - - - - - - - - - - - - - 
% outsource_startAppendix
%- - - - - - - - - - - - - - - - - - - - - - - - - - - - - - - - - - - - - - - - - - - - -
\makeatletter 
\def\@makechapterhead#1{
  {\parindent \z@ \raggedright \normalfont
    \interlinepenalty\@M
    \fontsize{16pt}{0pt}\bfseries Anhang \thechapter\quad #1\par\nobreak
    \vskip 60\p@
  }}
\makeatother
%- - - - - - - - - - - - - - - - - - - - - - - - - - - - - - - - - - - - - - - - - - - - -  % Nicht editieren!
%- - - - - - - - - - - - - - - - - - - - - - - - - - - - - - - - - - - - - - - - - - - - - - - - - - - - - - - - - - 

\anhangstuff % Generiert den Anhang-Titel und ändert Spezifikationen
{true} % ist Anhang vorhanden? (true=ja, false=nein)
{
% Anhang content
%
\chapter{Anhangstruktur} \label{refChapterAnhang}%
%
Hier sollte man am besten jegliche Teile über den \textbackslash inlcude-Befehl importieren. Die Überschriften werden genau gleich wie beim Hauptteil des Berichts über die Befehle \textbackslash chapter, \textbackslash section, \textbackslash subsection und \textbackslash subsubsection eingefügt. Die Layoutstruktur ist analog zu den normalen Kapiteln:%
%
\section{Unterkapitel im Anhang} \label{refSectionAnhang}%
%
\blindtext%
%
\subsection{Tieferes Kapitel}%
%
\subsubsection{Noch tieferes Kapitel}%
%
\blindtext%
%
%
%%
}

%- - - - - - - - - - - - - - - - - - - - - - - - - - - - - - - - - - - - - - - - - - - - - - - - - - - - - - - - - - 
% outsource_endAppendix
%- - - - - - - - - - - - - - - - - - - - - - - - - - - - - - - - - - - - - - - - - - - - -
\makeatletter 
\def\@makechapterhead#1{
  {\parindent \z@ \raggedright \normalfont
    \interlinepenalty\@M
    \fontsize{16pt}{0pt}\bfseries \thechapter\quad #1\par\nobreak
    \vskip 60\p@
  }}
\makeatother 
%- - - - - - - - - - - - - - - - - - - - - - - - - - - - - - - - - - - - - - - - - - - - - % Nicht editieren!
%- - - - - - - - - - - - - - - - - - - - - - - - - - - - - - - - - - - - - - - - - - - - - - - - - - - - - - - - - - 


%--------------------------------------------------------------------------------------------------------------------
% LITERATURVERZEICHNIS
%--------------------------------------------------------------------------------------------------------------------

\literaturverzeichnis%
{true} % LIteraturverzeichnis anzeigen? (true=ja, false=nein)


%--------------------------------------------------------------------------------------------------------------------
% BEZEICHNUNGEN
%--------------------------------------------------------------------------------------------------------------------

\bezeichnungenChapter%
{true} % Bezeichnungen anzeigen? (true=ja, false=nein)
{%b01_Bezeichnungen
%
\bezeichnungenSection{true}{Lateinische Grossbuchstaben}{
$A_c$ & Fläche eines Betonquerschnitts \\
$B$ & Belastungsgrad \\
$B_{cr}$ & Belastungsgrad bei Erreichen des Risslastniveaus\\
$E_c$ & Elastizitätsmodul von Beton \\
$M$ & Moment\\
P & Pol auf dem Mohrschen Kreis der Verzerrungen \\
$P$ & Einzellast\\
P$_{F}$ & Pol auf dem Mohrschen Kreis der aufgebrachten Spannungen \\
$Q$ & Last, Belastung\\
$RH$ & Luftfeuchtigkeit
}
%

\bezeichnungenSection{true}{Lateinische Kleinbuchstaben}{
%$c$ & Hilfsgrösse \\
$a_{s}$ & längenbezogener Bewehrungsquerschnitt\\
$c_u$, $c_o$ & Bewehrungsüberdeckung unten und oben \\
$c_{c\textrm{I}i\jj}$ & Ungerissene Betonsteifigkeitsmatrix \\
$c_{c\textrm{II}i\jj}$ & Gerissene Betonsteifigkeitsmatrix \\
$n_x$, $n_y$, $n_{xy}$ & Plattenschnittkräfte: Längenbezogene Normalkräfte\\
$q_x$, $q_y$, $q_z$ & Flächenlasten\\
$s$ & Beiwert Abbindegeschwindigkeit \\
$s_{rm}$ & diagonaler Rissabstand \\
$t_s$ & Zeitpunkt des Schwindbeginns \\
$u$ & Umfang des Betonquerschnitts \\
$x$, $y$, $z$ & Kartesische Koordinaten
}
%
\bezeichnungenSection{true}{Griechische Grossbuchstaben}{
$\Delta \sigma_{ci}$ & Tensor Änderung der Betonspannungen
}
%
\bezeichnungenSection{true}{Griechische Kleinbuchstaben}{
$\alpha$ & Faktor Abbindegeschwindigkeit, Drehwinkel Transformation\\
$\epsilon_{cs}$, $\epsilon_{csi}$ & Schwinddehnung bzw. Schwinddehnungstensor des Betons\\
$\epsilon_{cs,\infty}$ & Endschwindmass \\
$\rho_x$, $\rho_y$ & geometrischer Bewehrungsgehalt in $x$-Richtung bzw. in $y$-Richtung \\
$\varphi$  & Kriechzahl
}
%
\bezeichnungenSection{true}{Sonderzeichen}{
$\ee{\O}_x$, $\ee{\O}_y$ & Stabdurchmesser der Bewehrung in $x$-Richtung bzw. in $y$-Richtung\\%
$\partial$ & Differenz bei der partiellen Ableitung\\%
$\infty$ & unendlich%
}
%
\bezeichnungenSection{true}{Abkürzungen}{
CMM & Gerissenes Scheibenmodell \\
Emat & Steifigkeitsmatrix (Jakobimatrix) \\
GH & Modell für gerissene Hauptrichtungen \\
LE & Modell für linearelastisches Verhalten \\%
MC & Model Code
}}


%--------------------------------------------------------------------------------------------------------------------
% LEBENSLAUF (nur in Masterthesis)
%--------------------------------------------------------------------------------------------------------------------
%
%
% Lebenslauf
%
\lebenslauf{true}%
{% Personalien
Name & Peter Muster\\
Adresse & Bahnhofstrasse 1\newline
6004 Luzern\\%
Geburtsdatum & 01.01.1989\\
Heimatort & 6004 Luzern\\
Zivilstand & ledig%
}{% Ausbildung
August 1996 - Juli 2005 & Primar- und Sekundarschule, Dallenwil\\
August 2005 - Juli 2009 & Lehre als Bauzeichner mit technischer Berufsmaturität\newline Biegebruch GmbH, Luzern\\
September 2009 - Juli 2012 & Bauingenieurstudium Bachelor of Science\newline
Hochschule Luzern - Technik \& Architektur, Horw\\
September 2013 - Februar 2016 & Bauingenieurstudium Master of Science\newline
Vertiefung im Konstruktiven Ingenieurbau\newline
Hochschule Luzern - Technik \& Architektur, Horw
}{% Berufliche Tätigkeit
Juli 2010 - August 2010 & Bauzeichner bei Schubversagen AG, Luzern\\
Juli 2011 - August 2011 & Hilfsassistent Abteilung Bautechnik, \newline 
Hochschule Luzern - Technik \& Architektur, Horw \\
Dezember 2012 - September 2015 & Assistent Abteilung Bautechnik,\newline 
Hochschule Luzern - Technik \& Architektur, Horw%
}%
%
%
%--------------------------------------------------------------------------------------------------------------------
% TODO's (nur für Vorabzüge)
%--------------------------------------------------------------------------------------------------------------------

\listoftodos%     % Bei der definitiven Ausgabe des Dokuments auskommentieren

%- - - - - - - - - - - - - - - - - - - - - - - - - - - - - - - - - - - - - - - - - - - - - - - - - - - - - - - - - -  
\end{document} % Nicht editieren!
%- - - - - - - - - - - - - - - - - - - - - - - - - - - - - - - - - - - - - - - - - - - - - - - - - - - - - - - - - - 