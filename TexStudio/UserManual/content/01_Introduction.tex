% 01_Introduction
%
\spic{PreviousHwSetup.png}{Sensors and actors directly connected to a modem}{\label{fig:picPreviousSetup}}%
The exchange of data between on-board and off-board components is of vital importance with unmanned vehicles. Usually, each sensor or actor is directly connected to a modem (see \autoref{fig:picPreviousSetup})and dynamic data routing data between multiple devices and modems is not possible. When data transmission fails over one transmission technology, an intermediate platform is needed to detect this unreliable connection and quickly switch from one transmission technology to an other to ensure a stable data stream between vehicle and ground station.\\
\spic{NewHwSetup.png}{UAV Serial Switch as intermediate platform between devices and modems}{\label{fig:picNewSetup}}%
The UAV Serial Switch provides a solution to this problem. It is a platform for flexible data routing between devices and modems. It features four RS232 interfaces to connect data generating and processing devices (such as sensors and actors) and four RS232 interfaces to connect modems for data transmission. With jumpers, you either chose an USB connector or the RS232 interface as an input for each of the four serial connections available on device side. An possible use case can bee seen in \autoref{fig:picNewSetup} \\
\spic{PackageTransmission.png}{Assembling device data into packages}{\label{fig:picPackageTransmission}}%
The main functionality of this application is the sending of data packages. The software collects data from its devices, puts them into data packages with header, payload and CRC and sends those data packages out over the configured wireless serial connection. The corresponding second hardware receives the packages on its wireless side, extracts its payload and sends it out on the correct device side.\\
Acknowledges can be configured for each wireless connection. When a data package is received over this wireless connection, an acknowledges is generated and returned (see \autoref{fig:picPackageTransmission}). If the sender receives no acknowledges within the configured time, it will resend the data package. \\
The configuration file saved on the SD card leaves you in full control of the functionality of the Serial Switch. Modify it to change data routing between connected devices and modems, baud rate for each RS232 and USB connection and transmission behavior.\\
For details about hardware configurations, please read \autoref{sec:txtHwConfig} and for details about software configurations, please read \autoref{sec:txtSwConfig}.