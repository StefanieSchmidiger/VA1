\documentclass[a4paper,fleqn,english]{book}
%
%---------------------------------------------------------------
%  LaTeX - Template, Grundlagen, Tipps, Vorlagen (Version 09)
%  (c) April 2016, Manuel Wipfli, Stefan Lisibach
%  Hochschule Luzern - Technik und Architektur
%  Abteilung Bautechnik
%---------------------------------------------------------------
%
%
%
%--------------------------------------------------------------------------------------------------------------------
% PREAMBLE
%--------------------------------------------------------------------------------------------------------------------

\usepackage{corefiles/hsluBTmaster13}

\hypersetup{%
pdfcreator={pdflatex},
pdfproducer={LaTeX},
pdftitle={UAV Serial Switch},             %%% Titel der Arbeit UNBEDINGT ANPASSEN!
pdfsubject={VA1},                         %%% Thema (subject) UNBEDINGT ANPASSEN!
pdfauthor={Stefanie Schmidiger},          %%% Autor UNBEDINGT ANPASSEN!
pdfkeywords={Aeroscout},                  %%% Stichwörter UNBEDINGT ANPASSEN!
bookmarksnumbered=true,                   %%% Nummerierte Bookmarks
bookmarksopen=true,                       %%% Bookmarks bei PDF-öffnen bereits geöffnet?
colorlinks=false,                         %%% Farbig markierte Links
plainpages=false,                         %%% zur korrekten Erstellung der Bookmarks
bookmarksopenlevel=1,												 %%% Bookmarks nur bis Hierarchiestufe Section geöffnet
pdfpagelabels,                            %%% zur korrekten Erstellung der Bookmarks
hidelinks,                                %%% Links verstecken
pdfpagelayout=TwoPageRight                %%% Voreingestellte Ansicht im PDF-Editor (z.B. Acrobat)
}%

\graphicspath{{pictures/}}                %%% Pfad, wo die Bilder abgelegt werden
\bibliography{corefiles/literatur}        %%% Datei für die Literaturquellen

%\watermark{truefirstpage} % Wasserzeichen "Entwurf": (trueall, truefirstpage,false)


%--------------------------------------------------------------------------------------------------------------------
% DOKUMENT
%--------------------------------------------------------------------------------------------------------------------

%- - - - - - - - - - - - - - - - - - - - - - - - - - - - - - - - - - - - - - - - - - - - - - - - - - - - - - - - - - 
\begin{document}													% Nicht editieren!
\lsstyle                               % Ab hier Zeichenabstand +10 (Nicht editieren!)
\fontsize{10.5}{13.7}\selectfont       % In nachfolgenden Seiten Font 10.5pt (Nicht editieren!)
\pagenumbering{alph}                   % Nötig für Richtigkeit von backref-Verweisen (Nicht editieren!)
%- - - - - - - - - - - - - - - - - - - - - - - - - - - - - - - - - - - - - - - - - - - - - - - - - - - - - - - - - - 


%--------------------------------------------------------------------------------------------------------------------
% TITELBLATT, VERSIONSTABELLE UND SELBSTSTÄNDIGKEITSERKLÄRUNG
%--------------------------------------------------------------------------------------------------------------------

\prestuffmastershort                                 % Funkt. die TB, Versionstab. und Selbstst.-Erkl. generiert
{                                                    % 1. Input kann auch selber noch angepasst werden
%\huge\textbf{\LaTeX}\\                               %%% Titel der Arbeit 1. Zeile 
%\vspace{2mm}
\huge\textbf{UAV Serial Switch\\User Manual}\\     			     %%% Titel der Arbeit 2. Zeile
%\vspace{8mm}
%\Large\textbf{bla} %%% Untertitel der Arbeit
}
{Vertiefungsmodul I}                         	       %%% Art der Arbeit
{Stefanie Schmidiger}          	      		 			%%% Autor
{Prof. Erich Styger}                          	       		%%% Advisor
{Dr. Christian Vetterli}                                   		%%% Experte
{Horw}                                          		 	%%% Ort
{2018}                                         		 		%%% Jahr XXXX
{10.01.}                                       		 		%%% Tag und Monat der Selbstständigkeitserklärung XX.XX.
%{Version 0 & Initial Document & 10.01.18 & Stefanie Schmidiger}    	%%% Änderungsverzeichnis, für neue linie \\



%--------------------------------------------------------------------------------------------------------------------
% VORWORT
%--------------------------------------------------------------------------------------------------------------------

\vorwort%
{false} % ist Vorwort vorhanden? (true,false)
{% Vorwort
%
This work is being done for Aeroscout GmbH, a company that specialized in development of drones. \\
With unmanned vehicles, there are always on-board and off-board components. Data transmission between those components is of vital importance. Depending on the distance between on-board and off-board components, different data transmission technologies have to be used. \\
In this project, a hardware has been designed where multiple data inputs and outputs and multiple transmitters can be connected to a serial switch. The designed hardware features an SD card with a configuration file where data routing can be configured. \\
Data from connected devices will be collected and put into a data package with header, checksum, time stamp and other information. The package is then sent out via the configured transmitter. The corresponding second serial switch hardware receives this package, extracts and checks the payload, sends it out to the corresponding device and sends an acknowledge back to the package sender. \\
When data transmission over one transmission technology fails, the configuration file lets the user select the order of back up transmitters to be used. Data priority can also be configured because reliability of data transmission is extremely important with information such as exact location of the drone but not as important with information such as state of charge of the battery. \\
The serial switch hardware designed in the scope of this project features four serial RS232 connections where input and output devices can be connected that process or generate data. There are also four RS232 connectors where transmitters can be connected to send or receive data packages. The routing between data generating devices and transmitters to use can be done in a .ini file saved on an SD card. \\
There are two SPI to UART converters that act as the interface between the four devices connected and the micro controller respectively the four transmitters and the micro controller. \\
In a first version of the project, a Teensy 3.1 development board has been used as a micro controller unit. The software was written in the Arduino IDE with the provided Arduino libraries. As the project requirements became more complex, the limit of only a serial interface available as a debugging tool became more challenging. In the end, the first version of the software ran with more than ten tasks and an overhaul of the complex structure was necessary.\\
For this reason, an adapter board has been designed so the existing hardware could be used with the more powerful Teensy 3.5. This  adapter board features a SWD hardware debugging interface that was ready to use after removing a single component on the Teensy 3.5 development board. \\
The Teensy 3.5 was then configured to run with FreeRTOS. Task scheduler and queues provided by this operating system have been used to develop software that extracts data from received packages to output them on the configured interface or generates packages from received data bytes to send them out over the configured transmitter. The concept of acknowledges has also been applied so package loss can be detected and lost packages can be resent. \\
The software concept implemented is easy to understand, maintainable and expandable. Even though the functionality of the finished project remains the same as in the first version with Teensy 3.1 and Arduino, a refactoring has been necessary. Now further improvements and extra functionalities can be implemented more easily. } % Text des Vorworts
{Horw, January 2017} % Ort, Datum
{Stefanie Schmidiger} % Verfasser des Vorworts


%--------------------------------------------------------------------------------------------------------------------
% ZUSAMMENFASSUNG
%--------------------------------------------------------------------------------------------------------------------

\zusammenfassung%
{false} % Ist Abstract vorhanden?(truebothsamepage, truebothseparatepages, truedeutsch, false)
{% 02_Kurzfassung
%
With unmanned vehicles, there are always on-board and off-board components. Data transmission between those components is of vital importance. Depending on the distance between on-board and off-board components, different data transmission technologies have to be used. \\
In a previous project, the hardware for a Serial switch has been designed that features four RS232 interfaces to connect data processing and generating devices and four RS232 interfaces to connect modems for data transmission. The application running on the designed base board assembled data packages with the received data from its devices and sent those data packages out to the modems for transmission. The corresponding second Serial Switch received those data packages, checked them for validity and extracted the payload to send it out to its devices.\\
A Teensy 3.2 development board acted as the main micro controller. It is a small, inexpensive and powerful USB development board for Arduino applications. The software was flexible and in its header files the user could configure individual baud rates for each RS232 interface, data routing and the use of acknowledges for data packages for each modem side. The application was running with many tasks, complex and not easy to debug because of no hardware debugging interface.\\
Then this follow up project was initiated with the aim of an application with better maintainability and expandability. The requirement for this follow up project were the use of a more powerful micro controller with Free FROS as an operating system, the use of an SD card for a configuration file and data logging and a hardware debugging interface.\\
In the scope of this project, the Teensy 3.2 was replaced with a Teensy 3.5 development board, which featured an on-board SD card slot. The Teensy 3.5 was prepared for hardware debugging and an adapter board to use the new Teensy 3.5 with the on-board headers meant for the Teensy 3.2 was designed. This adapter board allowed the use of the same base board as was designed in the previous project.\\
For the Teensy 3.5 application, the concept with data packages is applied as well and the same configuration parameters are used. The configuration is read from an .ini file saved on the SD card.\\
The functionality of the application remains the same as in the Teensy 3.2 software but with better maintainability and an easier software concept with less tasks. Hardware debugging is now possible which is of vital importance for this application to be further expandable.\\
The Teensy 3.2 application was neither well documented nor running stable. While the Teensy 3.5 application provides the same functionalities as the previous software, all its issues are documented and possible workarounds are suggested. Data handling and data loss in case of unreliable data transmission channel is handled better and the application is running stable.} % Text der Kurzfassung in Deutsch
{} % Text der Kurzfassung in Englisch

%--------------------------------------------------------------------------------------------------------------------
% INHALTSVERZEICHNIS (automatisch generiert)
%--------------------------------------------------------------------------------------------------------------------

%- - - - - - - - - - - - - - - - - - - - - - - - - - - - - - - - - - - - - - - - - - - - - - - - - - - - - - - - - - 
% outsource_TOC
%- - - - - - - - - - - - - - - - - - - - - - - - - - - - - - - - - - - - - - - - - - - - -
\clearpage
\frontmatter      % Beginn der römischen Seitenzahlen
\pagestyle{fancy}       % Für nachfolgende Seiten
\markboth{Inhaltsverzeichnis}{}
\chapter*{Table of Contents} 
\pdfbookmark[1]{Inhaltsverzeichnis}{Inhaltsverzeichnis}
\vspace{-12.8mm}
\makeatletter
\@starttoc{toc}    % Generieren des Inhaltsverzeichnisses
\makeatother
%- - - - - - - - - - - - - - - - - - - - - - - - - - - - - - - - - - - - - - - - - - - - - 	% Nicht editieren!
%- - - - - - - - - - - - - - - - - - - - - - - - - - - - - - - - - - - - - - - - - - - - - - - - - - - - - - - - - - 


%--------------------------------------------------------------------------------------------------------------------
% BEGINN DER NUMMERIERTEN KAPITEL
%--------------------------------------------------------------------------------------------------------------------

%- - - - - - - - - - - - - - - - - - - - - - - - - - - - - - - - - - - - - - - - - - - - - - - - - - - - - - - - - - 
\mainmatter%         	% Nicht editieren!
\pagestyle{fancy}%	% Nicht editieren!
%- - - - - - - - - - - - - - - - - - - - - - - - - - - - - - - - - - - - - - - - - - - - - - - - - - - - - - - - - - 
\chapter{Introduction}
% 01_Introduction
%
\spic{PreviousHwSetup.png}{Sensors and actors directly connected to a modem}{\label{fig:picPreviousSetup}}%
The exchange of data between on-board and off-board components is of vital importance with unmanned vehicles. Usually, each sensor or actor is directly connected to a modem (see \autoref{fig:picPreviousSetup})and dynamic data routing data between multiple devices and modems is not possible. When data transmission fails over one transmission technology, an intermediate platform is needed to detect this unreliable connection and quickly switch from one transmission technology to an other to ensure a stable data stream between vehicle and ground station.\\
\spic{NewHwSetup.png}{UAV Serial Switch as intermediate platform between devices and modems}{\label{fig:picNewSetup}}%
The UAV Serial Switch provides a solution to this problem. It is a platform for flexible data routing between devices and modems. It features four RS232 interfaces to connect data generating and processing devices (such as sensors and actors) and four RS232 interfaces to connect modems for data transmission. With jumpers, you either chose an USB connector or the RS232 interface as an input for each of the four serial connections available on device side. An possible use case can bee seen in \autoref{fig:picNewSetup} \\
\spic{PackageTransmission.png}{Assembling device data into packages}{\label{fig:picPackageTransmission}}%
The main functionality of this application is the sending of data packages. The software collects data from its devices, puts them into data packages with header, payload and CRC and sends those data packages out over the configured wireless serial connection. The corresponding second hardware receives the packages on its wireless side, extracts its payload and sends it out on the correct device side.\\
Acknowledges can be configured for each wireless connection. When a data package is received over this wireless connection, an acknowledges is generated and returned (see \autoref{fig:picPackageTransmission}). If the sender receives no acknowledges within the configured time, it will resend the data package. \\
The configuration file saved on the SD card leaves you in full control of the functionality of the Serial Switch. Modify it to change data routing between connected devices and modems, baud rate for each RS232 and USB connection and transmission behavior.\\
For details about hardware configurations, please read \autoref{sec:txtHwConfig} and for details about software configurations, please read \autoref{sec:txtSwConfig}.
%
\chapter{Hardware Configuration}
% 02_HwConfiguration
%
\label{sec:txtHwConfig}
\spicv{BaseBoardAssembled.png}{UAV Serial Switch}{\label{fig:picBaseBoard}}{100}%
\spicv{BareBaseBoard.png}{Base board}{\label{fig:picBareBaseBoard}}{100}%
\spicv{Teensy35.png}{Teensy 3.5 development board}{\label{fig:picTeensy35}}{100}%
\spicv{AdapterBoard.png}{Teensy adapter board}{\label{fig:picBareAdapterBoard}}{100}%
The UAV Serial Switch can be seen in \autoref{fig:picBaseBoard} and aims to provide the user with maximum flexibility.\\
It consists of a base board (see \autoref{fig:picBareBaseBoard}), a Teensy adapter board (see \autoref{fig:picBareAdapterBoard}) and a Teensy 3.5 (see \autoref{fig:picTeensy35}).\\
The base board was designed for a Teensy 3.2 but is now used with a Teensy 3.5, hence the adapter board.\\
Read this chapter to learn more about the connectors and possible hardware configurations accessible to the user.
%
%
%
%
%
%
%
\section{User Interfaces}
\spicv{DeviceSideWirelessSide.png}{Device side and wireless side}{\label{fig:picDeviceSideWirelessSide}}{80}%
From now on, the side where data generating and data processing devices can be connected will be referred to as the device side and the side where modems can be connected will be referred to as the wireless side. See \autoref{fig:picDeviceSideWirelessSide} for the location of the connectors.
%
%
\subsection{RS232 Interface}
\spicv{Rs232Signals.png}{Pinout of RS232 user interfaces}{\label{fig:picRs232Pinout}}{100}%
There are four RS232 interfaces available for both device and wireless side. The sigals available are kept to a bare minimum with only RX, TX and ground lines. See \autoref{fig:picRs232Pinout} for details on pinout of the RS232 interfaces. All RS232 connectors on device side have the same pinout and all RS232 connectors on wireless side have the same pinout.\\
Connector numbering is also visible in \autoref{fig:picRs232Pinout}.
%
%
\subsection{USB Serial Port}
\spicv{UsbSignals.png}{USB user interfaces}{\label{fig:picUsbInterface}}{100}%
Instead of connecting devices on the RS232 interfaces, you can configure the Serial Switch for USB connected devices.\\
There are two USB connectors available on the base board. They act as dual USB to serial bridges, so each USB interface can replace two RS232 interfaces (see \autoref{fig:picUsbInterface}). When connecting one USB interface of the base board to a computer, two COM ports will appear to simulate two serial interfaces.\\
In order to use an USB interface instead of the RS232 connector for a device, the jumpers have to be set accordingly on the base board. See \autoref{fig:picUsbInterface} to find out which jumper corresponds to which serial interface. \\
Set the jumpers of the Tx and Rx signals to the left to select the RS232 signal and set it to the right to select the USB interface.\\
%
%
\section{Power Supply}
\spicv{PowerSupply.png}{Power supply options}{\label{fig:picPowerSupply}}{100}%
The base board and the adapter board share one power supply. The setup can either be powered by USB from any of the USB connectors or with an external 5V power supply (see \autoref{fig:picPowerSupply}). To select either one of those options, simply set the jumper to the correct position (External or USB, as indicated in  \autoref{fig:picPowerSupply}).
%
%
\section{UART Flow Control}
\spicv{UartFlowControl.png}{Hardware flow control signal}{\label{fig:picHwFlowControl}}{100}%
Because the RS232 connectors only route the RX and TX signals to the hardware buffer, a workaround has been done to use the CTS flow control signal as well for device connection 0 and 1.\\
To enable hardware flow control, connect the CTS signals of the RS232 interface to the indicated pin (see \autoref{fig:picHwFlowControl}). Careful: the CTS pin is not RS232 level compatible, you need a level shifter because the CTS pin only allows 0V to 5V!
%
%
%
\chapter{Software Configuration}
% 023_Introduction
%
\label{sec:txtSwConfig}
The Teensy 3.5 development board acts as the main micro controller. It collects data from the serial connections on device side, puts them into data packages with header and checksum and sends them out on the configured wireless side. When acknowledges are configured, the software will possibly resend the package if no acknowledge is received. The corresponding second hardware receives packages on its wireless side, extracts the payload and sends it out on its device side.\\
All communication is bidirectional, so data can be sent and received from all serial interfaces.\\
The Teensy has an on-board SD card slot with a configuration file on the micro SD card. All software configurations are read from that file upon power up of the Teensy. When modifying the config file, the Teensy has to be reset for the changes to take effect.\\
Possible configuration parameters can be seen in \autoref{tab:tabSwConfigParameters}:
%
\begin{center}
    \begin{longtable}{p{6cm}p{1cm}p{7cm}}
        \hline
        \textbf{Configuration parameter} & \textbf{Possible values} & \textbf{Description} \\
        \hline
        % --------------------------------------------------------
        BAUD\_RATES\_WIRELESS\_CONN & 9600, 38400, 57600, 115200 & 
        Baud rate to use on wireless side, configurable per wireless connection. Example: {9600, 38400, 57600, 115200} would result in 9600 baud for wireless connection 0, 38400 baud for wireless connection 1 etc.\\
        \hline
        % --------------------------------------------------------
        BAUD\_RATES\_DEVICE\_CONN &  9600, 38400, 57600, 115200 & 
        Baud rate to use on device side, configurable per cevice connection. Example: {9600, 38400, 57600, 115200} would result in 9600 baud for device connection 0, 38400 baud for device connection 1 etc.\\
        \hline
        % --------------------------------------------------------
        PRIO\_WIRELESS\_CONN\_DEV\_X &  0, 1, 2, 3, 4 & 
        This parameter determines over which wireless connection the data stream of a device will possibly be sent out. 0: this wireless connection will not be used. 1: Highest priority, data will be tried to send out over this connection first. 2: Second highest priority, data will be tried to send out over this connection should transmission over the first priority connection fail. 3: Third highest priority. 4: Lowest priority for data transmission. Example: {0, 2, 1, 0} would result in data being sent out over wireless connection 2 first and only sent out over wireless connection 1 in case of failure. All other wireless connections would not be used. Replace the X in the parameter name with 0, 1, 2 or 3.\\
        \hline
        % --------------------------------------------------------
        SEND\_CNT\_WIRELESS\_ CONN\_DEV\_X &  0...255 & 
        Determines how many times a package should tried to be sent out over a wireless connection before moving on to retrying with the next lower priority wireless connection. Example: {0, 5, 4, 0} would result in the package being sent out over wireless connection 1 five times and four times over wireless connection 2. Together with PRIO\_WIRELESS\_CONN\_DEV\_X, this parameter determines the number of resends per connection. Replace the X in the parameter name with 0, 1, 2 or 3.\\
        \hline
        % --------------------------------------------------------
        RESEND\_DELAY\_WIRELESS\_ CONN\_DEV\_X &  0...255 & 
        Determines how many milliseconds the software should wait for an acknowledge per wireless connection before sending the same package again. Example: {10, 0, 0, 0} would result in the software waiting for an acknowledge for 10ms when having sent a package out via wireless connection 0 before attempting a resend. Together with PRIO\_WIRELESS\_CONN\_DEV\_X, this parameter determines the delay of the resend behaviour Replace the X in the parameter name with 0, 1, 2 or 3.\\
        \hline
        % --------------------------------------------------------
        USUAL\_PACKET\_SIZE\_DEVICE\_ CONN  &  0...512 & 
        Maximum number of payload bytes per wireless package. 0: unknown payload-> the PACKAGE\_GEN\_MAX\_TIMEOUT parameter always determines the payload size. Example: {128, 0, 128, 128} results in a maximum payload of 128 bytes per package and an unknown maximum payload size for wireless connection 0.\\
        \hline
        % --------------------------------------------------------
        PACKAGE\_GEN\_MAX\_TIMEOUT  &  0...255 & 
        Maximum time (in milliseconds) that the software should wait for a package to fill up before sending it out anyway. Together with USUAL\_PACKET\_SIZE\_DEVICE\_CONN, this parameter determines the size of a package. Example: {50, 50, 50, 50} will result in data being sent out after a maximum wait time of 50ms.\\
        \hline
        % --------------------------------------------------------
        DELAY\_DISMISS\_OLD\_PACK\_ PER\_DEV &  0...255 & 
        Maximum time (in milliseconds) an old package should be tried to resend while the next package with data from the same device is available for sending. Example: {5, 5, 5, 5} results in a package being discarded 5ms after the next package is available in case it has not been sent successfully until then.\\
        \hline
        % --------------------------------------------------------
        SEND\_ACK\_PER\_WIRELESS\_CONN &  0, 1 & 
        Acknowledges turned on/off for each wireless connection. Example: {1, 1, 0, 0} results in acknowledges being expected and sent over wireless connection 0 and 1 but not over wireless connection 2 and 3.\\
        \hline
        % --------------------------------------------------------
        USE\_CTS\_PER\_WIRELESS\_CONN &  0, 1 & 
        Hardware flow control turned on/off for each wireless connection. Example: {1, 1, 0, 0} results in hardware flow control (CTS) for wireless connection 0 and 1 only. Currently, the hardware only supports CTS being enabled for these two connections as other CTS signals are not accessible to the user.\\
        \hline
        % --------------------------------------------------------
        TEST\_HW\_LOOPBACK\_ONLY & 0, 1 & 
        This parameter enables local echo. Any data received over any serial connection will be returned over the same connection immediately.\\
        \hline
        % --------------------------------------------------------
        GENERATE\_DEBUG\_OUTPUT &  0, 1 & 
        Information about the data throughput and other warnings will be printed out on the serial terminal.\\
        \hline
        % --------------------------------------------------------
        X\_TASK\_INTERVAL &  1 ... 65535 & 
        Task interval in milliseconds for each task. Replace X with the name of the task. Use a number greater than 2 to ensure that no task is blocking. \\
        \hline
        % --------------------------------------------------------
        \caption{Software configuration parameters}
        \label{tab:tabSwConfigParameters}    
    \end{longtable}
\end{center}
%
%
For task intervals, use a value greater than 2 milliseconds for SPI Handler, Package Handler and Network Handler. These three tasks provide the main functionality of the software as they do all the data processing (see documentation for details). Their task interval determines how frequently data is pulled from the devices and assembled to packages respectively extracted from packages and pushed out to the devices. The shell interval can be set to 50 milliseconds or even higher as it only determines how frequently the command interface is polled.\\ 
The configuration file needs to be named serialSwitch\_Config.ini\\
A sample configuration file can be seen below.\\
%
\begin{lstlisting}

;========================================================================================
[BaudRateConfiguration]
;
;
; BAUD_RATES_WIRELESS_CONN
; Configuration of baud rates on wireless side from 0 to 3.
; Regarding the supported baud rates see implementation of hwBufIfConfigureBaudRate in hwBufferInterface.cpp
BAUD_RATES_WIRELESS_CONN = 57600, 38400, 57600, 57600
;
;
; BAUD_RATES_DEVICE_CONN
; Configuration of baud rates on wireless side from 0 to 3.
; Regarding the supported baud rates see implementation of hwBufIfConfigureBaudRate in hwBufferInterface.cpp
BAUD_RATES_DEVICE_CONN = 57600, 57600, 38400, 38400
;
;
;========================================================================================
[ConnectionConfiguration]
;
;
; PRIO_WIRELESS_CONN_DEV_X
; Priority of the different wireless connections from the viewpoint of a single device.
; 0: Wireless connection is not used; 1: Highes priority; 2: Second priority, ..
PRIO_WIRELESS_CONN_DEV_0 = 1, 0, 0, 0
PRIO_WIRELESS_CONN_DEV_1 = 0, 1, 0, 0
PRIO_WIRELESS_CONN_DEV_2 = 0, 0, 1, 0
PRIO_WIRELESS_CONN_DEV_3 = 0, 0, 0, 1
;
;
; SEND_CNT_WIRELESS_CONN_DEV_X
; Number of times a package should be tried to be sent over a single wireless connection.
SEND_CNT_WIRELESS_CONN_DEV_0 = 1, 0, 0, 0
SEND_CNT_WIRELESS_CONN_DEV_1 = 0, 1, 0, 0
SEND_CNT_WIRELESS_CONN_DEV_2 = 0, 0, 1, 0
SEND_CNT_WIRELESS_CONN_DEV_3 = 0, 0, 0, 1
;
;
;========================================================================================
[TransmissionConfiguration]
;
;
; RESEND_DELAY_WIRELESS_CONN_DEV_X
; Time in ms that should be waited until a package is sent again when no acknowledge is 
; received per device and wireless connection.
RESEND_DELAY_WIRELESS_CONN_DEV_0 = 3, 3, 3, 3
RESEND_DELAY_WIRELESS_CONN_DEV_1 = 3, 3, 3, 3
RESEND_DELAY_WIRELESS_CONN_DEV_2 = 255, 255, 255, 255
RESEND_DELAY_WIRELESS_CONN_DEV_3 = 255, 255, 255, 255
;
;
; MAX_THROUGHPUT_WIRELESS_CONN
; Maximal throughput per wireless connection (0 to 3) in bytes/s.
MAX_THROUGHPUT_WIRELESS_CONN	 = 10000, 10000, 10000, 10000
;
;
; USUAL_PACKET_SIZE_DEVICE_CONN
; Usual packet size per device in bytes if known or 0 if unknown.
USUAL_PACKET_SIZE_DEVICE_CONN	= 25, 25, 25, 25
;
;
; PACKAGE_GEN_MAX_TIMEOUT
; Maximal time in ms that is waited until packet size is reached. If timeout is reached, 
; the packet will be sent anyway, independent of the amount of the available data.
PACKAGE_GEN_MAX_TIMEOUT	= 2, 2, 5, 5
;
;
; DELAY_DISMISS_OLD_PACK_PER_DEV
DELAY_DISMISS_OLD_PACK_PER_DEV	= 50, 50, 50, 50
;
;
; SEND_ACK_PER_WIRELESS_CONN
; To be able to configure on which wireless connections acknowledges should be sent if a 
; data package has been received. Set to 0 if no acknowledge should be sent, 1 if yes.
SEND_ACK_PER_WIRELESS_CONN	= 0, 1, 0, 0
;
;
; USE_CTS_PER_WIRELESS_CONN
; To be able to configure on which wireless connections CTS for hardware flow control 
; should be used. Set to 0 if it should not be used, 1 if yes.
; If enabled, data transmission is stopped CTS input is high and continued if low.
USE_CTS_PER_WIRELESS_CONN	= 0, 0, 0, 0
;
;
;========================================================================================
[SoftwareConfiguration]
;
;
; TEST_HW_LOOPBACK_ONLY
; Set to 0 for normal operation, 1 in order to enable loopback on all serial interfaces 
; in order to test the hardware.
TEST_HW_LOOPBACK_ONLY	= 0
;
; GENERATE_DEBUG_OUTPUT
; Set to 0 for normal operation, 1 in order to print out debug infos 
; (might be less performant).
GENERATE_DEBUG_OUTPUT	= 1;
;
; SPI_HANDLER_TASK_INTERVAL
; Interval in [ms] of corresponding task which he will be called. 0 would be no delay - 
; so to run as fast as possible.
SPI_HANDLER_TASK_INTERVAL	= 5;
;
; PACKAGE_GENERATOR_TASK_INTERVAL
; Interval in [ms] of corresponding task which he will be called. 0 would be no delay - 
; so to run as fast as possible.
PACKAGE_GENERATOR_TASK_INTERVAL	= 5;
;
; NETWORK_HANDLER_TASK_INTERVAL
; Interval in [ms] of corresponding task which he will be called. 0 would be no delay - 
; so to run as fast as possible.
NETWORK_HANDLER_TASK_INTERVAL	= 5;
;
; TOGGLE_GREEN_LED_INTERVAL
; Interval in [ms] in which the LED will be turned off or on -> frequency = 2x interval
TOGGLE_GREEN_LED_INTERVAL	= 500
;
; THROUGHPUT_PRINTOUT_TASK_INTERVAL
; Interval in [s] in which the throughput information will be printed out
THROUGHPUT_PRINTOUT_TASK_INTERVAL = 5
;
; SHELL_TASK_INTERVAL
; Interval in [ms] in which the shell task is called to refresh the shell 
; (which prints debug information and reads user inputs)
SHELL_TASK_INTERVAL		= 10
\end{lstlisting}
%
%
%
%
%
\section{Shell / Command Interface}
All debug information will be printed on a shell interface.\\
To view it, connect the debugger and either open the RTT Viewer or the RTT Client. Those applications can be found in the same folder as the Segger files have been installed, e.g. C:\textbackslash \textbackslash Program\textbackslash Segger\textbackslash
%--------------------------------------------------------------------------------------------------------------------
% LITERATURVERZEICHNIS
%-------------------------	-------------------------------------------------------------------------------------------

\literaturverzeichnis%
{false} % LIteraturverzeichnis anzeigen? (true=ja, false=nein)


%--------------------------------------------------------------------------------------------------------------------
% BEZEICHNUNGEN
%--------------------------------------------------------------------------------------------------------------------

\bezeichnungenChapter%
{true} % Bezeichnungen anzeigen? (true=ja, false=nein)
{%b01_Bezeichnungen
%
%
\bezeichnungenSection{true}{ }{
ALOHA & System for coordinating access to a shared communication channel \\
COM port & Simulated serial interface on computer \\
GND & Ground reference, usually 0V\\
HSLU & Lucerne University of Applied Sciences and Arts\\
HW & Hardware\\
ISO/OSI & 7 Layers Model \\
MCU & Micro Controller Unit \\
PC & Personal Computer\\
PCB & Printed Circuit Board\\
RS-232 & Serial interface with +-12V \\
RTT & Real Time Transfer, Segger terminal \\
RTOS & Realtime Operating System \\
RX & Received signal \\
SPI & Serial Peripheral Interface, synchronous communication standard\\
SW & Software\\
SWD & Serial Wire Debug, hardware debugging interface\\
TX & Transmitted signal \\
TTL & Transistor Transistor Level, 5V level\\
UART & Universal Asynchronous Receiver Transmitter \\
UAV & Unmanned Aeral Vehicle \\
USB & Universal Serial Bus
}}



%--------------------------------------------------------------------------------------------------------------------
% ANHANG
%--------------------------------------------------------------------------------------------------------------------

%- - - - - - - - - - - - - - - - - - - - - - - - - - - - - - - - - - - - - - - - - - - - - - - - - - - - - - - - - - 
% outsource_startAppendix
%- - - - - - - - - - - - - - - - - - - - - - - - - - - - - - - - - - - - - - - - - - - - -
\makeatletter 
\def\@makechapterhead#1{
  {\parindent \z@ \raggedright \normalfont
    \interlinepenalty\@M
    \fontsize{16pt}{0pt}\bfseries Anhang \thechapter\quad #1\par\nobreak
    \vskip 60\p@
  }}
\makeatother
%- - - - - - - - - - - - - - - - - - - - - - - - - - - - - - - - - - - - - - - - - - - - -  % Nicht editieren!
%- - - - - - - - - - - - - - - - - - - - - - - - - - - - - - - - - - - - - - - - - - - - - - - - - - - - - - - - - - 

\anhangstuff % Generiert den Anhang-Titel und ändert Spezifikationen
{false} % ist Anhang vorhanden? (true=ja, false=nein)
{
% Anhang content
%
\chapter{Anhangstruktur} \label{refChapterAnhang}%
%
Hier sollte man am besten jegliche Teile über den \textbackslash inlcude-Befehl importieren. Die Überschriften werden genau gleich wie beim Hauptteil des Berichts über die Befehle \textbackslash chapter, \textbackslash section, \textbackslash subsection und \textbackslash subsubsection eingefügt. Die Layoutstruktur ist analog zu den normalen Kapiteln:%
%
\section{Unterkapitel im Anhang} \label{refSectionAnhang}%
%
\blindtext%
%
\subsection{Tieferes Kapitel}%
%
\subsubsection{Noch tieferes Kapitel}%
%
\blindtext%
%
%
%%
}

%- - - - - - - - - - - - - - - - - - - - - - - - - - - - - - - - - - - - - - - - - - - - - - - - - - - - - - - - - - 
% outsource_endAppendix
%- - - - - - - - - - - - - - - - - - - - - - - - - - - - - - - - - - - - - - - - - - - - -
\makeatletter 
\def\@makechapterhead#1{
  {\parindent \z@ \raggedright \normalfont
    \interlinepenalty\@M
    \fontsize{16pt}{0pt}\bfseries \thechapter\quad #1\par\nobreak
    \vskip 60\p@
  }}
\makeatother 
%- - - - - - - - - - - - - - - - - - - - - - - - - - - - - - - - - - - - - - - - - - - - - % Nicht editieren!
%- - - - - - - - - - - - - - - - - - - - - - - - - - - - - - - - - - - - - - - - - - - - - - - - - - - - - - - - - - 

%--------------------------------------------------------------------------------------------------------------------
% TODO's (nur für Vorabzüge)
%--------------------------------------------------------------------------------------------------------------------

%\listoftodos%     % Bei der definitiven Ausgabe des Dokuments auskommentieren

%- - - - - - - - - - - - - - - - - - - - - - - - - - - - - - - - - - - - - - - - - - - - - - - - - - - - - - - - - -  
\end{document} % Nicht editieren!
%- - - - - - - - - - - - - - - - - - - - - - - - - - - - - - - - - - - - - - - - - - - - - - - - - - - - - - - - - - 