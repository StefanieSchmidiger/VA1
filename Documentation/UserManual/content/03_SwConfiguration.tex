% 023_Introduction
%
\label{sec:txtSwConfig}
The Teensy 3.5 development board acts as the main micro controller. It collects data from the serial connections on device side, puts them into data packages with header and checksum and sends them out on the configured wireless side. When acknowledges are configured, the software will possibly resend the package if no acknowledge is received. The corresponding second hardware receives packages on its wireless side, extracts the payload and sends it out on its device side.\\
All communication is bidirectional, so data can be sent and received from all serial interfaces.\\
The Teensy has an on-board SD card slot with a configuration file on the micro SD card. All software configurations are read from that file upon power up of the Teensy. When modifying the config file, the Teensy has to be reset for the changes to take effect.\\
Possible configuration parameters can be seen in \autoref{tab:tabSwConfigParameters}:
%
\begin{center}
    \begin{longtable}{p{6cm}p{1cm}p{7cm}}
        \hline
        \textbf{Configuration parameter} & \textbf{Possible values} & \textbf{Description} \\
        \hline
        % --------------------------------------------------------
        BAUD\_RATES\_WIRELESS\_CONN & 9600, 38400, 57600, 115200 & 
        Baud rate to use on wireless side, configurable per wireless connection. Example: {9600, 38400, 57600, 115200} would result in 9600 baud for wireless connection 0, 38400 baud for wireless connection 1 etc.\\
        \hline
        % --------------------------------------------------------
        BAUD\_RATES\_DEVICE\_CONN &  9600, 38400, 57600, 115200 & 
        Baud rate to use on device side, configurable per cevice connection. Example: {9600, 38400, 57600, 115200} would result in 9600 baud for device connection 0, 38400 baud for device connection 1 etc.\\
        \hline
        % --------------------------------------------------------
        PRIO\_WIRELESS\_CONN\_DEV\_X &  0, 1, 2, 3, 4 & 
        This parameter determines over which wireless connection the data stream of a device will possibly be sent out. 0: this wireless connection will not be used. 1: Highest priority, data will be tried to send out over this connection first. 2: Second highest priority, data will be tried to send out over this connection should transmission over the first priority connection fail. 3: Third highest priority. 4: Lowest priority for data transmission. Example: {0, 2, 1, 0} would result in data being sent out over wireless connection 2 first and only sent out over wireless connection 1 in case of failure. All other wireless connections would not be used. Replace the X in the parameter name with 0, 1, 2 or 3.\\
        \hline
        % --------------------------------------------------------
        SEND\_CNT\_WIRELESS\_ CONN\_DEV\_X &  0...255 & 
        Determines how many times a package should tried to be sent out over a wireless connection before moving on to retrying with the next lower priority wireless connection. Example: {0, 5, 4, 0} would result in the package being sent out over wireless connection 1 five times and four times over wireless connection 2. Together with PRIO\_WIRELESS\_CONN\_DEV\_X, this parameter determines the number of resends per connection. Replace the X in the parameter name with 0, 1, 2 or 3.\\
        \hline
        % --------------------------------------------------------
        RESEND\_DELAY\_WIRELESS\_ CONN\_DEV\_X &  0...255 & 
        Determines how many milliseconds the software should wait for an acknowledge per wireless connection before sending the same package again. Example: {10, 0, 0, 0} would result in the software waiting for an acknowledge for 10ms when having sent a package out via wireless connection 0 before attempting a resend. Together with PRIO\_WIRELESS\_CONN\_DEV\_X, this parameter determines the delay of the resend behaviour Replace the X in the parameter name with 0, 1, 2 or 3.\\
        \hline
        % --------------------------------------------------------
        USUAL\_PACKET\_SIZE\_DEVICE\_ CONN  &  0...512 & 
        Maximum number of payload bytes per wireless package. 0: unknown payload-> the PACKAGE\_GEN\_MAX\_TIMEOUT parameter always determines the payload size. Example: {128, 0, 128, 128} results in a maximum payload of 128 bytes per package and an unknown maximum payload size for wireless connection 0.\\
        \hline
        % --------------------------------------------------------
        PACKAGE\_GEN\_MAX\_TIMEOUT  &  0...255 & 
        Maximum time (in milliseconds) that the software should wait for a package to fill up before sending it out anyway. Together with USUAL\_PACKET\_SIZE\_DEVICE\_CONN, this parameter determines the size of a package. Example: {50, 50, 50, 50} will result in data being sent out after a maximum wait time of 50ms.\\
        \hline
        % --------------------------------------------------------
        DELAY\_DISMISS\_OLD\_PACK\_ PER\_DEV &  0...255 & 
        Maximum time (in milliseconds) an old package should be tried to resend while the next package with data from the same device is available for sending. Example: {5, 5, 5, 5} results in a package being discarded 5ms after the next package is available in case it has not been sent successfully until then.\\
        \hline
        % --------------------------------------------------------
        SEND\_ACK\_PER\_WIRELESS\_CONN &  0, 1 & 
        Acknowledges turned on/off for each wireless connection. Example: {1, 1, 0, 0} results in acknowledges being expected and sent over wireless connection 0 and 1 but not over wireless connection 2 and 3.\\
        \hline
        % --------------------------------------------------------
        USE\_CTS\_PER\_WIRELESS\_CONN &  0, 1 & 
        Hardware flow control turned on/off for each wireless connection. Example: {1, 1, 0, 0} results in hardware flow control (CTS) for wireless connection 0 and 1 only. Currently, the hardware only supports CTS being enabled for these two connections as other CTS signals are not accessible to the user.\\
        \hline
        % --------------------------------------------------------
        TEST\_HW\_LOOPBACK\_ONLY & 0, 1 & 
        This parameter enables local echo. Any data received over any serial connection will be returned over the same connection immediately.\\
        \hline
        % --------------------------------------------------------
        GENERATE\_DEBUG\_OUTPUT &  0, 1 & 
        Information about the data throughput and other warnings will be printed out on the serial terminal.\\
        \hline
        % --------------------------------------------------------
        X\_TASK\_INTERVAL &  1 ... 65535 & 
        Task interval in milliseconds for each task. Replace X with the name of the task. Use a number greater than 2 to ensure that no task is blocking. \\
        \hline
        % --------------------------------------------------------
        \caption{Software configuration parameters}
        \label{tab:tabSwConfigParameters}    
    \end{longtable}
\end{center}
%
%
For task intervals, use a value greater than 2 milliseconds for SPI Handler, Package Handler and Network Handler. These three tasks provide the main functionality of the software as they do all the data processing (see documentation for details). Their task interval determines how frequently data is pulled from the devices and assembled to packages respectively extracted from packages and pushed out to the devices. The shell interval can be set to 50 milliseconds or even higher as it only determines how frequently the command interface is polled.\\ 
The configuration file needs to be named serialSwitch\_Config.ini\\
A sample configuration file can be seen below.\\
%
\begin{lstlisting}

;========================================================================================
[BaudRateConfiguration]
;
;
; BAUD_RATES_WIRELESS_CONN
; Configuration of baud rates on wireless side from 0 to 3.
; Regarding the supported baud rates see implementation of hwBufIfConfigureBaudRate in hwBufferInterface.cpp
BAUD_RATES_WIRELESS_CONN = 57600, 38400, 57600, 57600
;
;
; BAUD_RATES_DEVICE_CONN
; Configuration of baud rates on wireless side from 0 to 3.
; Regarding the supported baud rates see implementation of hwBufIfConfigureBaudRate in hwBufferInterface.cpp
BAUD_RATES_DEVICE_CONN = 57600, 57600, 38400, 38400
;
;
;========================================================================================
[ConnectionConfiguration]
;
;
; PRIO_WIRELESS_CONN_DEV_X
; Priority of the different wireless connections from the viewpoint of a single device.
; 0: Wireless connection is not used; 1: Highes priority; 2: Second priority, ..
PRIO_WIRELESS_CONN_DEV_0 = 1, 0, 0, 0
PRIO_WIRELESS_CONN_DEV_1 = 0, 1, 0, 0
PRIO_WIRELESS_CONN_DEV_2 = 0, 0, 1, 0
PRIO_WIRELESS_CONN_DEV_3 = 0, 0, 0, 1
;
;
; SEND_CNT_WIRELESS_CONN_DEV_X
; Number of times a package should be tried to be sent over a single wireless connection.
SEND_CNT_WIRELESS_CONN_DEV_0 = 1, 0, 0, 0
SEND_CNT_WIRELESS_CONN_DEV_1 = 0, 1, 0, 0
SEND_CNT_WIRELESS_CONN_DEV_2 = 0, 0, 1, 0
SEND_CNT_WIRELESS_CONN_DEV_3 = 0, 0, 0, 1
;
;
;========================================================================================
[TransmissionConfiguration]
;
;
; RESEND_DELAY_WIRELESS_CONN_DEV_X
; Time in ms that should be waited until a package is sent again when no acknowledge is 
; received per device and wireless connection.
RESEND_DELAY_WIRELESS_CONN_DEV_0 = 3, 3, 3, 3
RESEND_DELAY_WIRELESS_CONN_DEV_1 = 3, 3, 3, 3
RESEND_DELAY_WIRELESS_CONN_DEV_2 = 255, 255, 255, 255
RESEND_DELAY_WIRELESS_CONN_DEV_3 = 255, 255, 255, 255
;
;
; MAX_THROUGHPUT_WIRELESS_CONN
; Maximal throughput per wireless connection (0 to 3) in bytes/s.
MAX_THROUGHPUT_WIRELESS_CONN	 = 10000, 10000, 10000, 10000
;
;
; USUAL_PACKET_SIZE_DEVICE_CONN
; Usual packet size per device in bytes if known or 0 if unknown.
USUAL_PACKET_SIZE_DEVICE_CONN	= 25, 25, 25, 25
;
;
; PACKAGE_GEN_MAX_TIMEOUT
; Maximal time in ms that is waited until packet size is reached. If timeout is reached, 
; the packet will be sent anyway, independent of the amount of the available data.
PACKAGE_GEN_MAX_TIMEOUT	= 2, 2, 5, 5
;
;
; DELAY_DISMISS_OLD_PACK_PER_DEV
DELAY_DISMISS_OLD_PACK_PER_DEV	= 50, 50, 50, 50
;
;
; SEND_ACK_PER_WIRELESS_CONN
; To be able to configure on which wireless connections acknowledges should be sent if a 
; data package has been received. Set to 0 if no acknowledge should be sent, 1 if yes.
SEND_ACK_PER_WIRELESS_CONN	= 0, 1, 0, 0
;
;
; USE_CTS_PER_WIRELESS_CONN
; To be able to configure on which wireless connections CTS for hardware flow control 
; should be used. Set to 0 if it should not be used, 1 if yes.
; If enabled, data transmission is stopped CTS input is high and continued if low.
USE_CTS_PER_WIRELESS_CONN	= 0, 0, 0, 0
;
;
;========================================================================================
[SoftwareConfiguration]
;
;
; TEST_HW_LOOPBACK_ONLY
; Set to 0 for normal operation, 1 in order to enable loopback on all serial interfaces 
; in order to test the hardware.
TEST_HW_LOOPBACK_ONLY	= 0
;
; GENERATE_DEBUG_OUTPUT
; Set to 0 for normal operation, 1 in order to print out debug infos 
; (might be less performant).
GENERATE_DEBUG_OUTPUT	= 1;
;
; SPI_HANDLER_TASK_INTERVAL
; Interval in [ms] of corresponding task which he will be called. 0 would be no delay - 
; so to run as fast as possible.
SPI_HANDLER_TASK_INTERVAL	= 5;
;
; PACKAGE_GENERATOR_TASK_INTERVAL
; Interval in [ms] of corresponding task which he will be called. 0 would be no delay - 
; so to run as fast as possible.
PACKAGE_GENERATOR_TASK_INTERVAL	= 5;
;
; NETWORK_HANDLER_TASK_INTERVAL
; Interval in [ms] of corresponding task which he will be called. 0 would be no delay - 
; so to run as fast as possible.
NETWORK_HANDLER_TASK_INTERVAL	= 5;
;
; TOGGLE_GREEN_LED_INTERVAL
; Interval in [ms] in which the LED will be turned off or on -> frequency = 2x interval
TOGGLE_GREEN_LED_INTERVAL	= 500
;
; THROUGHPUT_PRINTOUT_TASK_INTERVAL
; Interval in [s] in which the throughput information will be printed out
THROUGHPUT_PRINTOUT_TASK_INTERVAL = 5
;
; SHELL_TASK_INTERVAL
; Interval in [ms] in which the shell task is called to refresh the shell 
; (which prints debug information and reads user inputs)
SHELL_TASK_INTERVAL		= 10
\end{lstlisting}
%
%
%
%
%
\section{Shell / Command Interface}
All debug information will be printed on a shell interface.\\
To view it, connect the debugger and either open the RTT Viewer or the RTT Client. Those applications can be found in the same folder as the Segger files have been installed, e.g. C:\textbackslash \textbackslash Program\textbackslash Segger\textbackslash