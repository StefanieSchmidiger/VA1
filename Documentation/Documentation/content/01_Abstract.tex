% Vorwort
%
With unmanned vehicles, there are always on-board and off-board components. Data transmission between those components is of vital importance. Depending on the distance between vehicle and ground station, different data transmission technologies are ideal. So far, each device was connected to a single modem and the data transmission technology used could not be switched during operation.\\
In a previous project, a serial switch had been designed with four RS-232 interfaces that act as data input and output for devices and four RS-232 interfaces for transmitters and modems. This hardware is very flexible: data routing and transmission behavior is configurable by the user. The application running on the serial switch collects data from connected devices, puts it into a data package and sends it out via the configured transmitter. The corresponding second serial switch receives this package, extracts and verifies the payload, sends it out to the corresponding device and optionally sends an acknowledge back to the package sender.\\
A Teensy 3.2 development board has been used as a micro controller unit. The software was written in the Arduino IDE with the provided Arduino libraries. As the project requirements became more complex, the limit of only a serial interface available as a debugging tool became more challenging. In the end, the software ran with more than ten tasks and an overhaul of the complex structure was necessary.\\
This document describes the refactoring process of the previous project. In the scope of this work, an adapter board has been designed so the previous hardware could be used with the more powerful Teensy 3.5 development board and a hardware debugging interface. A new software concept for the Teensy 3.5 was developed and implemented.\\
The Teensy 3.5 is configured to run with FreeRTOS. The developed software uses the task scheduler and queues of the operating system to provide the same functionalities as the previous software for Teensy 3.2.\\
The new software concept for the Teensy 3.5 is easy to understand, maintainable and expandable. Even though the functionality of the finished project remains the same as in the first version with Teensy 3.2 and Arduino, a refactoring has been necessary. Now further improvements and extra functionalities can be implemented more easily as suggestions are given and issues are reported within this document.