%b01_Bezeichnungen
%
\bezeichnungenSection{true}{Lateinische Grossbuchstaben}{
$A_c$ & Fläche eines Betonquerschnitts \\
$B$ & Belastungsgrad \\
$B_{cr}$ & Belastungsgrad bei Erreichen des Risslastniveaus\\
$E_c$ & Elastizitätsmodul von Beton \\
$M$ & Moment\\
P & Pol auf dem Mohrschen Kreis der Verzerrungen \\
$P$ & Einzellast\\
P$_{F}$ & Pol auf dem Mohrschen Kreis der aufgebrachten Spannungen \\
$Q$ & Last, Belastung\\
$RH$ & Luftfeuchtigkeit
}
%

\bezeichnungenSection{true}{Lateinische Kleinbuchstaben}{
%$c$ & Hilfsgrösse \\
$a_{s}$ & längenbezogener Bewehrungsquerschnitt\\
$c_u$, $c_o$ & Bewehrungsüberdeckung unten und oben \\
$c_{c\textrm{I}i\jj}$ & Ungerissene Betonsteifigkeitsmatrix \\
$c_{c\textrm{II}i\jj}$ & Gerissene Betonsteifigkeitsmatrix \\
$n_x$, $n_y$, $n_{xy}$ & Plattenschnittkräfte: Längenbezogene Normalkräfte\\
$q_x$, $q_y$, $q_z$ & Flächenlasten\\
$s$ & Beiwert Abbindegeschwindigkeit \\
$s_{rm}$ & diagonaler Rissabstand \\
$t_s$ & Zeitpunkt des Schwindbeginns \\
$u$ & Umfang des Betonquerschnitts \\
$x$, $y$, $z$ & Kartesische Koordinaten
}
%
\bezeichnungenSection{true}{Griechische Grossbuchstaben}{
$\Delta \sigma_{ci}$ & Tensor Änderung der Betonspannungen
}
%
\bezeichnungenSection{true}{Griechische Kleinbuchstaben}{
$\alpha$ & Faktor Abbindegeschwindigkeit, Drehwinkel Transformation\\
$\epsilon_{cs}$, $\epsilon_{csi}$ & Schwinddehnung bzw. Schwinddehnungstensor des Betons\\
$\epsilon_{cs,\infty}$ & Endschwindmass \\
$\rho_x$, $\rho_y$ & geometrischer Bewehrungsgehalt in $x$-Richtung bzw. in $y$-Richtung \\
$\varphi$  & Kriechzahl
}
%
\bezeichnungenSection{true}{Sonderzeichen}{
$\ee{\O}_x$, $\ee{\O}_y$ & Stabdurchmesser der Bewehrung in $x$-Richtung bzw. in $y$-Richtung\\%
$\partial$ & Differenz bei der partiellen Ableitung\\%
$\infty$ & unendlich%
}
%
\bezeichnungenSection{true}{Abkürzungen}{
CMM & Gerissenes Scheibenmodell \\
Emat & Steifigkeitsmatrix (Jakobimatrix) \\
GH & Modell für gerissene Hauptrichtungen \\
LE & Modell für linearelastisches Verhalten \\%
MC & Model Code
}