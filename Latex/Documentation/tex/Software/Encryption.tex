\section*{Encryption}
\subsection*{General information about encryption}
The purpose of encrypting data is to secure information so that only the intended receiver can decypher and read it. It is not entirely impossible to decrypt a encrypted message without the encryption key but considerable computational resources would be required.
In encryption, the original data or message, in cryptography referred to as plaintext. There are different algorithms that can be used for encryption and all of them require an encryption key. The encrypted plaintext is then referred to as ciphertext. The ciphertext can be decrypted easily by using the intended key.\\
There are two techniques to ensure confidentiality: symmetric and asymmetric encryption.\\

\subsection*{Confidentiality: symmetric and asymmetric encryption}
In symmetric encryption, both parties are in possession of the same key and that key us used for encryption and decryption. This is the oldest and best-known technique and was often used in letters. The key then consisted of a word or a number and was applied to the message in a particular way such as shifting the alphabet in a particular order. With the knowledge of the encryption key, the message could be deciphered easily. One profound issue with symmetric encryption is the transmission of the shared key to the other party without it falling into the wrong hands.\\
In asymmetric encryption, there are two separate keys: a private key and a public key. When encrypting a message with a public key, it can only be decrypted with the corresponding private key and vice versa. This solves the problem of having to secretly transmit the key to the receiver because the key that is intended for the opposite party is a public key. There is no way to compute the private key when knowing the public key. So for two users to communicate with encrypted messages, they would have to agree on an encryption algorithm and use the other users public key to encrypt a message. Both users can send a query over the network to receive the other users public key. The disadvantage of the asymmetric encryption is that it requires far more processing power than the symmetric encryption.\\
Encryption ensures the confidentiality of a message but other measurements are still required to ensure integrity and authenticity.\\

\subsection*{Integrity}
There are two types of data integrity threats: passive and active threads.\\
Passive threads are due to accidental change of data such as noise on the transmission channel or data getting corrupted upon saving it. The simplest way to detect those errors is by adding a error-correcting code or a checksum.\\
In an active thread, a third party tampers with the original message and might even replace the error-correcting code or checksum with a newly calculated one. One way to prevent it is by using hash functions. Hash functions map arbitrary sized data to data of fixed size (called hash codes, digests or simply hashes). By only changing one bit of the original data, the hash codes will change almost completely.\\

\subsection*{Authenticity}
Authenticity proves that the message originated from a specific entity. This can be done by adding an authentication tag. When the message is altered with, the authentication tag becomes invalid.\\