\section*{Software validity}

\textbf{CRC vs Checksum}\\
Anytime data is stored in a computer with the intent to transmit it, there is a need to ensure that the data is not corrupted. If corrupted data was sent, there would be inaccurate data transmitted and  it may not work as desired. There is, therefore, a need for an error detection system that checks that all the data entered is okay and not corrupt before any encryption or transmission occurs. There are two main methods to check the data.\\
Checksum is arguably the oldest methods that has been used in the validation of all data prior to its being sent. Checksum also helps in authenticating data, as the raw data and the entered data should conform. If an anomaly is noticed, referred to as an invalid checksum, there is a suggestion that there may have been a data compromise in a given method.\\
Cyclic redundancy check, or CRC as it is commonly referred to, is a concept also employed in the validation of data. The principle used by CRC is similar to checksums, but rather than use the 8 byte system employed by Checksum in checking for data consistency, polynomial division is used in the determination of the CRC. The CRC is most commonly 16 or 32 bits in length. If a single byte is missing, an inconsistency is flagged in the data as it does not add up to the original.\\
\textbf{Differences}\\
One of the differences noted between the 2 is that CRC employs a math formula that is based on 16- or 32-bit encoding as opposed to Checksum that is based on 8 bytes in checking for data anomalies. The CRC is based on a hash approach while Checksum gets its values from an addition of all truncated data which may come in 8 or 16 bits. CRC, therefore, has a greater ability to recognize data errors as a single bit missing in the hash system which changes the overall result.\\
The checksum, on the other hand, requires less transparency and will provide for ample error detection as it employs an addition of bytes with the variable. It can, therefore, be said that the main purpose of CRC is to catch a diverse range of errors that may come about during the transmission of data in analog mode. Checksum, on the other hand, can be said to have been designed for the sole purpose of noting regular errors that may occur during software implementation.\\
CRC is an improvement over checksums. As earlier noted, checksums are a traditional form of computing, and CRC’s are just a mere advancement of the arithmetic that increases the complexity of the computation.  This, in essence, increases the available patterns that are present, and thus more errors can be detected by the method. Checksum has been shown to detect mainly single-bit errors. However, CRC can detect any double-bit errors being observed in the data computation. In understanding the differences between the two data validation methods, knowledge is gathered as to why these two methods are used hand-in-hand in Internet protocol, as it reduces the vulnerability of Internet protocols occurring.\\
If you enable ECC (which is highly recommended), then the data rate you can support is halved. The ECC system doubles the size of the data sent by the radios. It is worth it however, as the bit error rate will drop dramatically, and you are likely to get a much more reliable link at longer ranges.\\
Read more: \url{Differences Between CRC And Checksum | Difference Between http://www.differencebetween.net/technology/software-technology/differences-between-crc-and-checksum/#ixzz4trxUtflQ} \\

\subsection*{Error correction in Modem}
When using the RFD868x model, a variety of configurations can be made that will affect data throughput and errors.\\
The ECC (Error Correcting Code) parameter makes a big difference to the data rate you can support at a given AIR\_SPEED. If you have ECC set to zero, then no error correcting information is sent, and the radio uses a simple 16 bit CRC to detect transmission errors. In that case your radio will be able to support data transfers in one direction of around 90\% of the AIR\_SPEED.\\
Read more: \url{http://ardupilot.org/copter/docs/common-3dr-radio-advanced-configuration-and-technical-information.html#common-3dr-radio-advanced-configuration-and-technical-information}

\subsection*{CRC}
Header: 11 Bytes -> 10 Data bytes + 1 byte CRC (CRC8). CRC8 > correction up to 256 bit possible
Payload: max 150 (normally 50...100 bytes). 2 byte CRC -> Fehlerwahrscheinlichkeit?