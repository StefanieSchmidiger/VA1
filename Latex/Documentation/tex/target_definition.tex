\section*{Target definition}
Version: 1 \\
Edit date: 20.09.2017 \\ \\
Below the official, initial task descriptions are stated as written in [VM1]. After that each point is addressed separately in a subsection, where the points are described in more detail.

\subsection*{Official problem definition}
\begin{itemize}
	\item[a)] Evaluierung des Genauigkeit/Rechenzeit-Kompromisses für die in [3][4] erwähnten Beispielanwendungen Sobel-Kantenfilter und FFT, im Vergleich zu reinen SW- bzw. HW-Implementierungen
	\item[b)] Identifizierung weiterer potentieller Effizienzkriterien beim Einsatz neuronaler Beschleuniger in SoC-Umgebungen
	\item[c)] Definition, Training und Simulation eines (entsprechend vereinfachten) neuronalen Netzwerkes für die beiden Beispielanwendungen Sobel-Kantenfilter und FFT in der TensorFlow-Umgebung, siehe [7]
	\item[d)] Definition weitergehender, für das Kompetenzzentrum ISN relevanter Forschungsfragen bzw. Applikationen zum Thema Approximate Computing (evtl. auch ohne neuronale Beschleuniger)
\end{itemize}

\subsection*{Task a) - FFT}
Search and evaluate current implementations of the FFT application:
\begin{itemize}
	\item \textbf{Software:} Research in general, use of data from CC ISN?, www.FFTW.org
	\item \textbf{Approximation:} Research in general on Approximation with a neural network / neural acceleration
	\item \textbf{Hardware:} Use data from VM2 from [Elia]
\end{itemize}
There are no own measurements planned on this topic, but task c) will pick up the results of this task. Furthermore the evaluation will be on the following three topics:
\begin{itemize}
	\item \textbf{Accuracy:} In percentage. What about SNR? Test bench from Elia?
	\item \textbf{Speed:} In absolute time. Use case?
	\item \textbf{HW cost:} Is this even a possible measurement?
\end{itemize}
\textbf{Deliverable:} Range/Datapoints of accuracy achievable with SW/Approximation/HW in a plot. X-axis is SW/Approximation/HW, y-axis is accuracy in percentage.

\subsection*{Task a) - Sobol edge detection}
Search and evaluate current implementations of the Sobol edge detection application:
\begin{itemize}
	\item \textbf{Software:} Use data from CC ISN - Vision group - [Eric Ropraz] Master Thesis, also research
	\item \textbf{Approximation:} Research in general on Approximation with a neural network / neural acceleration
	\item \textbf{Hardware:} Use data from CC ISN - Vision group - [Eric Ropraz] Master Thesis, also research
\end{itemize}
There are no own measurements planned on this topic, but task c) will pick up the results of this task. Furthermore the evaluation will be on the following three topics:
\begin{itemize}
	\item \textbf{Accuracy:} ?
	\item \textbf{Speed:} ?
	\item \textbf{HW cost:} ?
\end{itemize}
\textbf{Deliverable:} Range/Datapoints of accuracy achievable with SW/Approximation/HW in a plot. X-axis is SW/Approximation/HW, y-axis is accuracy in percentage.

\subsection*{Task b)}
\textbf{Deliverable:} List of possible efficiency criteria with a rational for each one.

\subsection*{Task c)}
\textbf{Deliverables:}
\begin{itemize}
	\item Code for an NN which does an FFT. Input parameters are to be defined
	\item Code for an NN which does a Sobol edge detection. Input parameters are to be defined
	\item Code to train each NN
	\item Simulation results of NN on use case X and Y (to be defined) and comparison to A and B (to be defined)
\end{itemize}

\subsection*{Task d)}
\textbf{Deliverables:}
\begin{itemize}
	\item List of possible research questions which could be interesting from the viewpoint of the CC ISN
	\item List of possible applications for AC with NN or AP in general in current CC ISN projects
\end{itemize}